%% bare_jrnl.tex
%% V1.3
%% 2007/01/11
%% by Michael Shell
%% see http://www.michaelshell.org/
%% for current contact information.
%%
%% This is a skeleton file demonstrating the use of IEEEtran.cls
%% (requires IEEEtran.cls version 1.7 or later) with an IEEE journal paper.
%%
%% Support sites:
%% http://www.michaelshell.org/tex/ieeetran/
%% http://www.ctan.org/tex-archive/macros/latex/contrib/IEEEtran/
%% and
%% http://www.ieee.org/



% *** Authors should verify (and, if needed, correct) their LaTeX system  ***
% *** with the testflow diagnostic prior to trusting their LaTeX platform ***
% *** with production work. IEEE's font choices can trigger bugs that do  ***
% *** not appear when using other class files.                            ***
% The testflow support page is at:
% http://www.michaelshell.org/tex/testflow/


%%*************************************************************************
%% Legal Notice:
%% This code is offered as-is without any warranty either expressed or
%% implied; without even the implied warranty of MERCHANTABILITY or
%% FITNESS FOR A PARTICULAR PURPOSE! 
%% User assumes all risk.
%% In no event shall IEEE or any contributor to this code be liable for
%% any damages or losses, including, but not limited to, incidental,
%% consequential, or any other damages, resulting from the use or misuse
%% of any information contained here.
%%
%% All comments are the opinions of their respective authors and are not
%% necessarily endorsed by the IEEE.
%%
%% This work is distributed under the LaTeX Project Public License (LPPL)
%% ( http://www.latex-project.org/ ) version 1.3, and may be freely used,
%% distributed and modified. A copy of the LPPL, version 1.3, is included
%% in the base LaTeX documentation of all distributions of LaTeX released
%% 2003/12/01 or later.
%% Retain all contribution notices and credits.
%% ** Modified files should be clearly indicated as such, including  **
%% ** renaming them and changing author support contact information. **
%%
%% File list of work: IEEEtran.cls, IEEEtran_HOWTO.pdf, bare_adv.tex,
%%                    bare_conf.tex, bare_jrnl.tex, bare_jrnl_compsoc.tex
%%*************************************************************************

% Note that the a4paper option is mainly intended so that authors in
% countries using A4 can easily print to A4 and see how their papers will
% look in print - the typesetting of the document will not typically be
% affected with changes in paper size (but the bottom and side margins will).
% Use the testflow package mentioned above to verify correct handling of
% both paper sizes by the user's LaTeX system.
%
% Also note that the "draftcls" or "draftclsnofoot", not "draft", option
% should be used if it is desired that the figures are to be displayed in
% draft mode.
%
\documentclass[journal]{IEEEtran}
%
% If IEEEtran.cls has not been installed into the LaTeX system files,
% manually specify the path to it like:
% \documentclass[journal]{../sty/IEEEtran}





% Some very useful LaTeX packages include:
% (uncomment the ones you want to load)


% *** MISC UTILITY PACKAGES ***
%
%\usepackage{ifpdf}
% Heiko Oberdiek's ifpdf.sty is very useful if you need conditional
% compilation based on whether the output is pdf or dvi.
% usage:
% \ifpdf
%   % pdf code
% \else
%   % dvi code
% \fi
% The latest version of ifpdf.sty can be obtained from:
% http://www.ctan.org/tex-archive/macros/latex/contrib/oberdiek/
% Also, note that IEEEtran.cls V1.7 and later provides a builtin
% \ifCLASSINFOpdf conditional that works the same way.
% When switching from latex to pdflatex and vice-versa, the compiler may
% have to be run twice to clear warning/error messages.






% *** CITATION PACKAGES ***
%
%\usepackage{cite}
% cite.sty was written by Donald Arseneau
% V1.6 and later of IEEEtran pre-defines the format of the cite.sty package
% \cite{} output to follow that of IEEE. Loading the cite package will
% result in citation numbers being automatically sorted and properly
% "compressed/ranged". e.g., [1], [9], [2], [7], [5], [6] without using
% cite.sty will become [1], [2], [5]--[7], [9] using cite.sty. cite.sty's
% \cite will automatically add leading space, if needed. Use cite.sty's
% noadjust option (cite.sty V3.8 and later) if you want to turn this off.
% cite.sty is already installed on most LaTeX systems. Be sure and use
% version 4.0 (2003-05-27) and later if using hyperref.sty. cite.sty does
% not currently provide for hyperlinked citations.
% The latest version can be obtained at:
% http://www.ctan.org/tex-archive/macros/latex/contrib/cite/
% The documentation is contained in the cite.sty file itself.






% *** GRAPHICS RELATED PACKAGES ***
%
\ifCLASSINFOpdf
  \usepackage[pdftex]{graphicx}
  % declare the path(s) where your graphic files are
  % \graphicspath{{../pdf/}{../jpeg/}}
  % and their extensions so you won't have to specify these with
  % every instance of \includegraphics
  \DeclareGraphicsExtensions{.pdf}
  \DeclareGraphicsExtensions{.pdf,.jpeg,.png}
\else
  % or other class option (dvipsone, dvipdf, if not using dvips). graphicx
  % will default to the driver specified in the system graphics.cfg if no
  % driver is specified.
  \usepackage[dvips]{graphicx}
  % declare the path(s) where your graphic files are
  % \graphicspath{{../eps/}}
  % and their extensions so you won't have to specify these with
  % every instance of \includegraphics
  \DeclareGraphicsExtensions{.eps}
\fi
% graphicx was written by David Carlisle and Sebastian Rahtz. It is
% required if you want graphics, photos, etc. graphicx.sty is already
% installed on most LaTeX systems. The latest version and documentation can
% be obtained at: 
% http://www.ctan.org/tex-archive/macros/latex/required/graphics/
% Another good source of documentation is "Using Imported Graphics in
% LaTeX2e" by Keith Reckdahl which can be found as epslatex.ps or
% epslatex.pdf at: http://www.ctan.org/tex-archive/info/
%
% latex, and pdflatex in dvi mode, support graphics in encapsulated
% postscript (.eps) format. pdflatex in pdf mode supports graphics
% in .pdf, .jpeg, .png and .mps (metapost) formats. Users should ensure
% that all non-photo figures use a vector format (.eps, .pdf, .mps) and
% not a bitmapped formats (.jpeg, .png). IEEE frowns on bitmapped formats
% which can result in "jaggedy"/blurry rendering of lines and letters as
% well as large increases in file sizes.
%
% You can find documentation about the pdfTeX application at:
% http://www.tug.org/applications/pdftex





% *** MATH PACKAGES ***
%
%\usepackage[cmex10]{amsmath}
% A popular package from the American Mathematical Society that provides
% many useful and powerful commands for dealing with mathematics. If using
% it, be sure to load this package with the cmex10 option to ensure that
% only type 1 fonts will utilized at all point sizes. Without this option,
% it is possible that some math symbols, particularly those within
% footnotes, will be rendered in bitmap form which will result in a
% document that can not be IEEE Xplore compliant!
%
% Also, note that the amsmath package sets \interdisplaylinepenalty to 10000
% thus preventing page breaks from occurring within multiline equations. Use:
%\interdisplaylinepenalty=2500
% after loading amsmath to restore such page breaks as IEEEtran.cls normally
% does. amsmath.sty is already installed on most LaTeX systems. The latest
% version and documentation can be obtained at:
% http://www.ctan.org/tex-archive/macros/latex/required/amslatex/math/





% *** SPECIALIZED LIST PACKAGES ***
%
%\usepackage{algorithmic}
% algorithmic.sty was written by Peter Williams and Rogerio Brito.
% This package provides an algorithmic environment fo describing algorithms.
% You can use the algorithmic environment in-text or within a figure
% environment to provide for a floating algorithm. Do NOT use the algorithm
% floating environment provided by algorithm.sty (by the same authors) or
% algorithm2e.sty (by Christophe Fiorio) as IEEE does not use dedicated
% algorithm float types and packages that provide these will not provide
% correct IEEE style captions. The latest version and documentation of
% algorithmic.sty can be obtained at:
% http://www.ctan.org/tex-archive/macros/latex/contrib/algorithms/
% There is also a support site at:
% http://algorithms.berlios.de/index.html
% Also of interest may be the (relatively newer and more customizable)
% algorithmicx.sty package by Szasz Janos:
% http://www.ctan.org/tex-archive/macros/latex/contrib/algorithmicx/




% *** ALIGNMENT PACKAGES ***
%
%\usepackage{array}
% Frank Mittelbach's and David Carlisle's array.sty patches and improves
% the standard LaTeX2e array and tabular environments to provide better
% appearance and additional user controls. As the default LaTeX2e table
% generation code is lacking to the point of almost being broken with
% respect to the quality of the end results, all users are strongly
% advised to use an enhanced (at the very least that provided by array.sty)
% set of table tools. array.sty is already installed on most systems. The
% latest version and documentation can be obtained at:
% http://www.ctan.org/tex-archive/macros/latex/required/tools/


%\usepackage{mdwmath}
%\usepackage{mdwtab}
% Also highly recommended is Mark Wooding's extremely powerful MDW tools,
% especially mdwmath.sty and mdwtab.sty which are used to format equations
% and tables, respectively. The MDWtools set is already installed on most
% LaTeX systems. The lastest version and documentation is available at:
% http://www.ctan.org/tex-archive/macros/latex/contrib/mdwtools/


% IEEEtran contains the IEEEeqnarray family of commands that can be used to
% generate multiline equations as well as matrices, tables, etc., of high
% quality.


%\usepackage{eqparbox}
% Also of notable interest is Scott Pakin's eqparbox package for creating
% (automatically sized) equal width boxes - aka "natural width parboxes".
% Available at:
% http://www.ctan.org/tex-archive/macros/latex/contrib/eqparbox/





% *** SUBFIGURE PACKAGES ***
%\usepackage[tight,footnotesize]{subfigure}
% subfigure.sty was written by Steven Douglas Cochran. This package makes it
% easy to put subfigures in your figures. e.g., "Figure 1a and 1b". For IEEE
% work, it is a good idea to load it with the tight package option to reduce
% the amount of white space around the subfigures. subfigure.sty is already
% installed on most LaTeX systems. The latest version and documentation can
% be obtained at:
% http://www.ctan.org/tex-archive/obsolete/macros/latex/contrib/subfigure/
% subfigure.sty has been superceeded by subfig.sty.



%\usepackage[caption=false]{caption}
%\usepackage[font=footnotesize]{subfig}
% subfig.sty, also written by Steven Douglas Cochran, is the modern
% replacement for subfigure.sty. However, subfig.sty requires and
% automatically loads Axel Sommerfeldt's caption.sty which will override
% IEEEtran.cls handling of captions and this will result in nonIEEE style
% figure/table captions. To prevent this problem, be sure and preload
% caption.sty with its "caption=false" package option. This is will preserve
% IEEEtran.cls handing of captions. Version 1.3 (2005/06/28) and later 
% (recommended due to many improvements over 1.2) of subfig.sty supports
% the caption=false option directly:
%\usepackage[caption=false,font=footnotesize]{subfig}
%
% The latest version and documentation can be obtained at:
% http://www.ctan.org/tex-archive/macros/latex/contrib/subfig/
% The latest version and documentation of caption.sty can be obtained at:
% http://www.ctan.org/tex-archive/macros/latex/contrib/caption/




% *** FLOAT PACKAGES ***
%
%\usepackage{fixltx2e}
% fixltx2e, the successor to the earlier fix2col.sty, was written by
% Frank Mittelbach and David Carlisle. This package corrects a few problems
% in the LaTeX2e kernel, the most notable of which is that in current
% LaTeX2e releases, the ordering of single and double column floats is not
% guaranteed to be preserved. Thus, an unpatched LaTeX2e can allow a
% single column figure to be placed prior to an earlier double column
% figure. The latest version and documentation can be found at:
% http://www.ctan.org/tex-archive/macros/latex/base/



%\usepackage{stfloats}
% stfloats.sty was written by Sigitas Tolusis. This package gives LaTeX2e
% the ability to do double column floats at the bottom of the page as well
% as the top. (e.g., "\begin{figure*}[!b]" is not normally possible in
% LaTeX2e). It also provides a command:
%\fnbelowfloat
% to enable the placement of footnotes below bottom floats (the standard
% LaTeX2e kernel puts them above bottom floats). This is an invasive package
% which rewrites many portions of the LaTeX2e float routines. It may not work
% with other packages that modify the LaTeX2e float routines. The latest
% version and documentation can be obtained at:
% http://www.ctan.org/tex-archive/macros/latex/contrib/sttools/
% Documentation is contained in the stfloats.sty comments as well as in the
% presfull.pdf file. Do not use the stfloats baselinefloat ability as IEEE
% does not allow \baselineskip to stretch. Authors submitting work to the
% IEEE should note that IEEE rarely uses double column equations and
% that authors should try to avoid such use. Do not be tempted to use the
% cuted.sty or midfloat.sty packages (also by Sigitas Tolusis) as IEEE does
% not format its papers in such ways.


%\ifCLASSOPTIONcaptionsoff
%  \usepackage[nomarkers]{endfloat}
% \let\MYoriglatexcaption\caption
% \renewcommand{\caption}[2][\relax]{\MYoriglatexcaption[#2]{#2}}
%\fi
% endfloat.sty was written by James Darrell McCauley and Jeff Goldberg.
% This package may be useful when used in conjunction with IEEEtran.cls'
% captionsoff option. Some IEEE journals/societies require that submissions
% have lists of figures/tables at the end of the paper and that
% figures/tables without any captions are placed on a page by themselves at
% the end of the document. If needed, the draftcls IEEEtran class option or
% \CLASSINPUTbaselinestretch interface can be used to increase the line
% spacing as well. Be sure and use the nomarkers option of endfloat to
% prevent endfloat from "marking" where the figures would have been placed
% in the text. The two hack lines of code above are a slight modification of
% that suggested by in the endfloat docs (section 8.3.1) to ensure that
% the full captions always appear in the list of figures/tables - even if
% the user used the short optional argument of \caption[]{}.
% IEEE papers do not typically make use of \caption[]'s optional argument,
% so this should not be an issue. A similar trick can be used to disable
% captions of packages such as subfig.sty that lack options to turn off
% the subcaptions:
% For subfig.sty:
% \let\MYorigsubfloat\subfloat
% \renewcommand{\subfloat}[2][\relax]{\MYorigsubfloat[]{#2}}
% For subfigure.sty:
% \let\MYorigsubfigure\subfigure
% \renewcommand{\subfigure}[2][\relax]{\MYorigsubfigure[]{#2}}
% However, the above trick will not work if both optional arguments of
% the \subfloat/subfig command are used. Furthermore, there needs to be a
% description of each subfigure *somewhere* and endfloat does not add
% subfigure captions to its list of figures. Thus, the best approach is to
% avoid the use of subfigure captions (many IEEE journals avoid them anyway)
% and instead reference/explain all the subfigures within the main caption.
% The latest version of endfloat.sty and its documentation can obtained at:
% http://www.ctan.org/tex-archive/macros/latex/contrib/endfloat/
%
% The IEEEtran \ifCLASSOPTIONcaptionsoff conditional can also be used
% later in the document, say, to conditionally put the References on a 
% page by themselves.





% *** PDF, URL AND HYPERLINK PACKAGES ***
%
%\usepackage{url}
% url.sty was written by Donald Arseneau. It provides better support for
% handling and breaking URLs. url.sty is already installed on most LaTeX
% systems. The latest version can be obtained at:
% http://www.ctan.org/tex-archive/macros/latex/contrib/misc/
% Read the url.sty source comments for usage information. Basically,
% \url{my_url_here}.


\usepackage{eurosym}



% *** Do not adjust lengths that control margins, column widths, etc. ***
% *** Do not use packages that alter fonts (such as pslatex).         ***
% There should be no need to do such things with IEEEtran.cls V1.6 and later.
% (Unless specifically asked to do so by the journal or conference you plan
% to submit to, of course. )


% correct bad hyphenation here
\hyphenation{op-tical net-works semi-conduc-tor}


\begin{document}
%
% paper title
% can use linebreaks \\ within to get better formatting as desired
\title{Overview of the LOFAR signal processing architecture}
%
%
% author names and IEEE memberships
% note positions of commas and nonbreaking spaces ( ~ ) LaTeX will not break
% a structure at a ~ so this keeps an author's name from being broken across
% two lines.
% use \thanks{} to gain access to the first footnote area
% a separate \thanks must be used for each paragraph as LaTeX2e's \thanks
% was not built to handle multiple paragraphs
%

\author{Andr\'{e} W. Gunst,
        Ronald Nijboer,
        and~John W. Romein% <-this % stops a space
\thanks{A. Gunst, R. Nijboer, and J.W. Romein are with ASTRON, Dwingeloo, 
The Netherlands, www.astron.nl
}% <-this % stops a space
\thanks{Manuscript received January 31, 2008; revised .}}

% note the % following the last \IEEEmembership and also \thanks - 
% these prevent an unwanted space from occurring between the last author name
% and the end of the author line. i.e., if you had this:
% 
% \author{....lastname \thanks{...} \thanks{...} }
%                     ^------------^------------^----Do not want these spaces!
%
% a space would be appended to the last name and could cause every name on that
% line to be shifted left slightly. This is one of those "LaTeX things". For
% instance, "\textbf{A} \textbf{B}" will typeset as "A B" not "AB". To get
% "AB" then you have to do: "\textbf{A}\textbf{B}"
% \thanks is no different in this regard, so shield the last } of each \thanks
% that ends a line with a % and do not let a space in before the next \thanks.
% Spaces after \IEEEmembership other than the last one are OK (and needed) as
% you are supposed to have spaces between the names. For what it is worth,
% this is a minor point as most people would not even notice if the said evil
% space somehow managed to creep in.



% The paper headers
\markboth{Journal of \LaTeX\ Class Files,~Vol.~6, No.~1, January~2007}%
{Shell \MakeLowercase{\textit{et al.}}: Bare Demo of IEEEtran.cls for Journals}
% The only time the second header will appear is for the odd numbered pages
% after the title page when using the twoside option.
% 
% *** Note that you probably will NOT want to include the author's ***
% *** name in the headers of peer review papers.                   ***
% You can use \ifCLASSOPTIONpeerreview for conditional compilation here if
% you desire.




% If you want to put a publisher's ID mark on the page you can do it like
% this:
%\IEEEpubid{0000--0000/00\$00.00~\copyright~2007 IEEE}
% Remember, if you use this you must call \IEEEpubidadjcol in the second
% column for its text to clear the IEEEpubid mark.



% use for special paper notices
%\IEEEspecialpapernotice{(Invited Paper)}


\newcommand{\fixme}[1]{{\bf\em #1}}


% make the title area
\maketitle


\begin{abstract}
%\boldmath
LOFAR is the first of a new generation of phased-array radio telescopes,
that combines the signals from many thousands of simple, omni-directional
antennas, rather than from expensive dishes.
Its revolutionary design and unprecedented size enables observations in the
hardly-explored 10--250~MHz frequency range, and allows the study of
a vast amount of new science cases.

This paper describes the LOFAR signal processing chain from the stations,
where the signals are received, via the Wide Area Network to the Central 
Processors. The Central processing is split in real-time correlation and 
off-line calibration and imaging.
\end{abstract}
% IEEEtran.cls defaults to using nonbold math in the Abstract.
% This preserves the distinction between vectors and scalars. However,
% if the journal you are submitting to favors bold math in the abstract,
% then you can use LaTeX's standard command \boldmath at the very start
% of the abstract to achieve this. Many IEEE journals frown on math
% in the abstract anyway.

% Note that keywords are not normally used for peerreview papers.
%\begin{IEEEkeywords}
%IEEEtran, journal, \LaTeX, paper, template.
%\end{IEEEkeywords}






% For peer review papers, you can put extra information on the cover
% page as needed:
% \ifCLASSOPTIONpeerreview
% \begin{center} \bfseries EDICS Category: 3-BBND \end{center}
% \fi
%
% For peerreview papers, this IEEEtran command inserts a page break and
% creates the second title. It will be ignored for other modes.
\IEEEpeerreviewmaketitle



\section{Introduction}
% The very first letter is a 2 line initial drop letter followed
% by the rest of the first word in caps.
% 
% form to use if the first word consists of a single letter:
% \IEEEPARstart{A}{demo} file is ....
% 
% form to use if you need the single drop letter followed by
% normal text (unknown if ever used by IEEE):
% \IEEEPARstart{A}{}demo file is ....
% 
% Some journals put the first two words in caps:
% \IEEEPARstart{T}{his demo} file is ....
% 
% Here we have the typical use of a "T" for an initial drop letter
% and "HIS" in caps to complete the first word.
\IEEEPARstart{T}{his} demo file is intended to serve as a ``starter file''
for IEEE journal papers produced under \LaTeX\ using
IEEEtran.cls version 1.7 and later.
% You must have at least 2 lines in the paragraph with the drop letter
% (should never be an issue)
I wish you the best of success.

In the Netherlands a LO(w) Frequency ARray (LOFAR) is developed optimized for the low frequency band from 30 - 240 MHz. LOFAR is the first large scale radio telescope based fully on the phased array technique. This prevents the use of moving large constructions and enables multi-beaming. The total collecting area of LOFAR is achieved by many small dipoles which are grouped in stations to reduce the data rate to an acceptable level. The main reduction is achieved by selecting only a part of the sky by using the phased array technique. The number of stations installed in the Netherlands will be at least 36. Half of the number of stations will be core stations and the other half remote stations. The main difference between them is that the core stations can be split up in two independent arrays delivering the two fold of the remote station bandwidth.

All data of the stations is transported to one central place via a Wide Area Network (WAN) and using owned and leased light paths. In Groningen all station data is processed. The processing is done by off the shelf hardware, varying from a supercomputer to clusters of computers. The processing done will vary from correlation for the standard imaging mode to tied-array beamforming for the transients and pulsar mode for example. However, also more sophisticated processing will be done to enable various energy cosmic ray modes of different intensities. 

The heart of LOFAR will be installed in the Northern part of the Netherlands. The maximum baseline of LOFAR including only the Dutch stations is about 100 km. Since, also other European institutes showed interest and are building LOFAR stations the maximal baseline of LOFAR is extended to at least xxx km.  

In this paper the LOFAR architecture will be discussed with a focus on the data flow.

\hfill mds
 
\hfill January 11, 2007

\subsection{Subsection Heading Here}
Subsection text here.

% needed in second column of first page if using \IEEEpubid
%\IEEEpubidadjcol

\subsubsection{Subsubsection Heading Here}
Subsubsection text here.


% An example of a floating figure using the graphicx package.
% Note that \label must occur AFTER (or within) \caption.
% For figures, \caption should occur after the \includegraphics.
% Note that IEEEtran v1.7 and later has special internal code that
% is designed to preserve the operation of \label within \caption
% even when the captionsoff option is in effect. However, because
% of issues like this, it may be the safest practice to put all your
% \label just after \caption rather than within \caption{}.
%
% Reminder: the "draftcls" or "draftclsnofoot", not "draft", class
% option should be used if it is desired that the figures are to be
% displayed while in draft mode.
%
%\begin{figure}[!t]
%\centering
%\includegraphics[width=2.5in]{myfigure}
% where an .eps filename suffix will be assumed under latex, 
% and a .pdf suffix will be assumed for pdflatex; or what has been declared
% via \DeclareGraphicsExtensions.
%\caption{Simulation Results}
%\label{fig_sim}
%\end{figure}

% Note that IEEE typically puts floats only at the top, even when this
% results in a large percentage of a column being occupied by floats.


% An example of a double column floating figure using two subfigures.
% (The subfig.sty package must be loaded for this to work.)
% The subfigure \label commands are set within each subfloat command, the
% \label for the overall figure must come after \caption.
% \hfil must be used as a separator to get equal spacing.
% The subfigure.sty package works much the same way, except \subfigure is
% used instead of \subfloat.
%
%\begin{figure*}[!t]
%\centerline{\subfloat[Case I]\includegraphics[width=2.5in]{subfigcase1}%
%\label{fig_first_case}}
%\hfil
%\subfloat[Case II]{\includegraphics[width=2.5in]{subfigcase2}%
%\label{fig_second_case}}}
%\caption{Simulation results}
%\label{fig_sim}
%\end{figure*}
%
% Note that often IEEE papers with subfigures do not employ subfigure
% captions (using the optional argument to \subfloat), but instead will
% reference/describe all of them (a), (b), etc., within the main caption.


% An example of a floating table. Note that, for IEEE style tables, the 
% \caption command should come BEFORE the table. Table text will default to
% \footnotesize as IEEE normally uses this smaller font for tables.
% The \label must come after \caption as always.
%
%\begin{table}[!t]
%% increase table row spacing, adjust to taste
%\renewcommand{\arraystretch}{1.3}
% if using array.sty, it might be a good idea to tweak the value of
% \extrarowheight as needed to properly center the text within the cells
%\caption{An Example of a Table}
%\label{table_example}
%\centering
%% Some packages, such as MDW tools, offer better commands for making tables
%% than the plain LaTeX2e tabular which is used here.
%\begin{tabular}{|c||c|}
%\hline
%One & Two\\
%\hline
%Three & Four\\
%\hline
%\end{tabular}
%\end{table}


% Note that IEEE does not put floats in the very first column - or typically
% anywhere on the first page for that matter. Also, in-text middle ("here")
% positioning is not used. Most IEEE journals use top floats exclusively.
% Note that, LaTeX2e, unlike IEEE journals, places footnotes above bottom
% floats. This can be corrected via the \fnbelowfloat command of the
% stfloats package.

\section{Station processing}

In the LOFAR stations the electromagnetic signals of interest are received by multiple dipoles and combined in such a way that only a part of the sky is selected. The main architecture is depicted in Figure~\ref{fig:stationarch}. 

%fig:stationarch LOFAR remote station architecture

\subsection{Antennas}

The remote station baseline design consists of 96 Low Band High (LBH) antennas optimized for the 30-80 MHz frequency range and 48 High Band compound Antenna arrays (HBA) optimized for the 120-240 MHz frequency range. One HBA is composed of 16 individual antenna elements, which are combined through analog beamforming with true time delays. Both the low band and combined high band antenna signals are pre-filtered and amplified near the antenna prior to transportation over coaxial cables to a central location within a station. The size of the low band array will be 70xxx m in diameter, while the HBA array will be more dense. 

\subsection{Receiver}
The receiver accomodates also for a third antenna input (Low Band Low) operating in the 10-30 MHz band. At the central station location the receiver unit selects one out of three antennas. After selecting an antenna, the signal is filtered with one of the integrated filters. These filters select one of the four available observing bands. After filtering, the signal is amplified and filtered again to reduce the out of band noise contribution. A pre-amplifier in front of the A/D converter converts the single ended signal into a differential signal prior to A/D conversion. For the receiver a wideband direct digital conversion architecture is adopted. This reduces the number of analog devices used in the signal path. Since the LOFAR stations are installed in civilized areas the dynamic range of the A/D converter must be sufficient to handle the Radio Frequency Interference (RFI) signals in the band of interest. Hence, the A/D converter converts the analog signal into a 12 bit digital signal. The maximum sampling rate is 200 MHz, which is sufficient to directly convert the analog signals preventing the use of analog mixers. To fill the gaps in between the Nyquist zones, a sample frequency of 160 MHz can be chosen as well. The Nyquist zones I to III of the A/D converter with a sample frequency of 200 MHz and 160 MHz respectively are depicted in~\ref{fig:nyquistzones}. 

%fig:nyquistzones caption: The supported modes in the LOFAR receiver based on the available Nyquist zones

\subsection{Digital Processing}

The antennas in the station form a phased array, producing one or many station beams on the sky. Multi-beaming is a major advantage of the phased array concept. It is not only used to increase observational efficiency, but may be vital for calibration purposes. 

To form a phased array at station level, the analog antenna signals must be delayed and added to form a beam on the sky. Moreover the beamformer should be able to track sources on the sky and be flexible in exchanging beams for bandwidth. 

The beamformer can be implemented by using true time delays or by applying phase shifts on narrow subbands. For the last solution an error is made at the edges of each subband (Figure~\ref{fig:???}), since the phase is frequency dependent and only one phase can be set per subband. 

%fig:phasebeamf caption: Illustration of 4 subbands and the error which is introduced by approximating the time delays by phase shifts per subband (the black line on the right hand side is the ideal phase)

Since, the correlator in the LOFAR system is an FX correlator a certain frequency resolution is required for that. For the beamformer a certain (other) frequency resolution is required as well (if implemented by phase shifts). This led to the decision to implement the station beamforming with phase shifts after a first stage filterbank which realizes a frequency resolution sufficient for the beamforming operation. A second stage filterbank will make an even higher frequency resolution which is required before the correlator. Another reason to split the filter banks up is that the first stage filter bank is more expensive than the second stage filter bank because the first stage filter bank operates per antenna while the second stage filter bank operates on beams. Since no extra significant data reduction will be done after the second stage filterbank, that functionality will be implemented in the central systems.

The first stage filter bank in the stations splits up the total band into 512 equidistant subbands. After the filtering operation, specific subbands can be selected. The selected subbands can be arbitrary and will add up to in total 32~MHz. This bandwidth is matched to the current capacity of the central processor.

To form beams, the antenna signals are combined in a complex weighted sum. This is done with independent beamformers for each subband. In this way the number of pointings on the sky can be exchanged against the bandwidth per pointing, i.e. a user can choose between 1 beam of 32~MHz to a maximal of 8 beams of 4~MHz.

%Additionally the station processing is able to calculate the full cross correlation matrix of all dipoles in the station for one subband. This information is used for the station calibration strategy. The goal of the station calibration is to calibrate individual antenna signals for gain and phase differences. 

In parallel with the filtering and beamforming the raw data can be stored in a transient buffer. Freezing of the buffer content can be controlled by internal or external triggers. The stored data or selections thereof can be sent to the central processing facility for further processing. This extension is primarily used for the transients and cosmic ray community.
%referenties naar LBA paper, Stefan papers, ...
\fixme{Say something about UDP packets, 16+16 bit samples, max 54 subband/RSP board}

\section{Central Processing: the Correlator}

The station data are centrally received and combined in the
correlator~\cite{Romein:06}.
LOFAR uses an IBM Blue Gene/L supercomputer to correlate all data,
unlike traditional telescopes that typically use customized hardware.
The desire for a flexible and reconfigurable instrument demands a
{\em software\/} solution, but the data rates and processing requirements
demand a supercomputer.

\fixme{Explain: BG/L characteristics}
The heart of the Central Processor is a six-rack IBM Blue Gene/L supercomputer
that provides 34~TFLOPS peak performance and has a fast internal interconnect:
the 3-D~torus.
A total of 768~Gigabit Ethernet interfaces are available for external I/O.
Each processor is extended by two double-precision floating point units
that are very suitable for signal processing, since a variety of operations
on complex numbers are natively supported.


\ldots

\begin{figure*}
\includegraphics[width=\textwidth]{flow}
\caption{Real-time filters on the Central Processor.  For simplicity, the
figure shows two stations, one subband, one polarization.  The numbers are
valid for the 200~MHz mode.}
\label{fig:flow}
\end{figure*}

\fixme{Explain: input buffer \& station synchronization, transpose, delay
compenstation, PPF, correlator}

The input section receives all station data that are sent via the wide-area
network.
% I assume that the UDP packets have been introduced before
Since UDP is an unreliable protocol, the input section handles duplicated,
lost, and out-of-order packets.
Lost data is appropriately flagged as being "invalid", and the remainder
of the processing pipeline takes this into account.
The data are received into a circular buffer, that holds the most recent
six seconds of data.

The buffer serves three purposes.
First, it synchronizes the station data, since the difference in path lengths
between the stations and the correlator result in different travel times over
the WAN link.
Second, the buffer is used to compensate for the bulk of the delay that
is due to the fact that the stations receive a wave at a different time
(see Figure~\ref{fig:delay}).
%Figure~\ref{fig:delay}), and the bulk of the delay is compensated for
%by delaying the stream of station samples for each station differently.
%The delay depends on the physical locations of the stations, the observation
%direction, and the time.
%The rotation of the earth complicates these computations, since the orientation
%of the stations with respect to the observed sky is altered continuously.
% TODO: explain delays better
Third, it provides some headroom to recover from small hiccups in the remainder
of the processing pipeline, without data loss.

Each input node receives data from up to 54~subbands from a single RSP board
of one station.
Unfortunately, this data distribution is not suitable for the correlator,
because to correlate a subband, the data from all stations are needed.
Also, we need hundreds of CPUs to correlate 54~subbands.
The next step is to redistribute all data over the CPUs, using a fast
interconnect.
Initially, we did this on a separate input cluster connected by a high-speed
infiniband network, but we found that the internal 3-D~torus network within
the Blue Gene/L can do this job much faster.

\ldots

Next in the pipeline is the second-stage PolyPhase (PPF) filter, that splits
each 195~KHz (resp.\ 156~KHz) subband into 256~channels of 763~Hz
(resp.\ 610~Hz) wide.
Splitting the subbands into narrow frequency channels allows flagging
of narrow-band RFI without much data loss (see Section~{sec:RFI}).
The PPF filter consists of 256~Finite Impulse Filters (FIR) filters and a
Fast Fourier Transform (see Figure~\ref{fig:flow}).
Each FIR filter is a 16-tap band pass filter.
The incoming samples are round-robin distributed over the FIR filters;
the outputs are fourier transformed.
To achieve optimal performance, both the FIR filter and the FFT are implemented
in assembly (although we maintain an equivalent C++ version for debugging
and portability purposes).

After the subbands are split into narrow channels, the remainder of the
delays are compensated, by shifting the phase of each sample.
The correction factor depends on time and frequency.
The delays are computed exactly for the beginning and ending of each
integration period (typically: one second), and interpolated both in time
and (channel) frequency such that the phase of each sample is corrected by
an accurate factor.



The PPF implicitly converts the 16-bit complex number to 32-bit floating-point
numbers, since the BG/L performs floating point computations faster than
integer operations.
Also, the phase correction and the correlations are done in floating point.



\ldots

\fixme{Maybe this paragraph should be moved to the IEEE Computer paper.}
An example that illustrates the benefits of a flexible software solution was
the ease with which we could move the second-stage PolyPhase Filter, which was
originally designed to run on FPGAs at the stations, to the Blue Gene/L.
Once we saw that we could obtain really high performance from the BG/L in
practice and recognized that sufficient computational power was left,
a considerable cost reduction was achieved by removing the PPF FPGAs from the
station design.

\ldots

Although the BG/L is computationally very efficient,
streaming the station data into the machine at the required data rates
turned out to be a major problem,
despite the BG/L's atypical high number of I/O interfaces.
For each 16~BG/L compute cores, there is one {\em I/O~node\/} that
has one external gigabit-Ethernet interface and transparently handles all I/O
calls initiated by its associated compute cores (see Figure~\ref{fig:IOnode}).
We found that the stock network system software was not particularly optimized
for high-throughput I/O, and that the obtained bandwidths were far from what
theoretically should be achievable.

The dissatisfaction about the performance and about the I/O model in general
led to a joint effort to redesign the entire network software infrastructure,
and resulted in a new environment called {\em ZOID\/}~\cite{Iskra:08}.
ZOID does not only yield better performance, but it is much more flexible,
since it allows application code to be run on the I/O~node.
With ZOID, we were able to move the receipt of the station data from the input
cluster nodes to the BG/L I/O~nodes, so that the station data are sent
directly through the WAN into the BG/L.
Not having to build a separate input cluster results in an estimated cost
saving of \euro700,000.

\section{Central Processing: Calibration}

The processing of LOFAR data has to deal with a number of challenges~\cite{Noordam:04,Nijboer:07}. First of all, the data volumes are huge and being able to process the data in finite time dictates the design of the data processing chain. Second, compared to traditional steel dishes, the phased array station beams are far more variable (in time, in frequency, as well as over the different stations), they yield a higher degree of instrumental polarization, and they have relatively high sidelobes. All these issues complicate the processing of the data. Especially, since a high dynamic range must be reached. The third category of challenges lies in the sky itself. At the low frequencies where LOFAR will observe there will be very bright sources so that a high dynamic range and, hence, a high accuracy is needed to see the faint background sources. The sky will also be filled with a large number of sources, giving rise to confusion. Finally, the Earths ionosphere seriously defocusses the images.

With LOFAR we enter a new regime in radio astronomical data processing.  The challenges imply that for LOFAR we have to reconsider exisiting processing strategies and algorithms and develop new strategies and algorithms. The processing therefore remains a work in progress of which we give an overview of the current status.

\subsection{Processing large data volumes}

The total amount of data that is produced is detemined by the total number of stations that are used in the observation. This number is still uncertain and will be different in the Low Band and the High Band, since the High Band stations in the Core will be split in 2 half stations. Using 20 Core stations and 20 Remote stations as a working example this generates between 0.28 Gbyte/s for the LBA Core Array in 200 MHz sampling mode and 3.1 Gbyte/s for the HBA Full Array in 160 MHz sampling mode. After a typical observation of 4 hours between 3.3 Tbyte and 35 Tbyte will be collected.

Since a permanent data storage is not part of the LOFAR telescope these data volumes have to be processed near real time. Fortunately, not all LOFAR applications are so data intense, so that for every 1 hour of observation we may have up to, say, 4 hours to further process the data. With this in mind data I/O becomes a serious problem. Obviously the data needs to be processed in a parallelized and distributed way minimizing the I/O that is needed~\cite{Loose:08,Diepen:08}.  
 
Data can be distributed over a large number of processing nodes in a number of ways. Distribution over baselines is not very suitable for imaging, where data from all baselines must be combined to produce an image. Distribution over time has the disadvantage that up to sevaral Gbytes/s have to be sent to a single processing node. Frequency, therefore, seems to be the best way. This distribution scheme matches with the design of the correlator. It is also a convenient scheme for the imager, where images are created per (combined) frequency channel. 

A consequence of distribution over frequency is that in the self-calibration step solver equations from different compute nodes may need to be combined. The combining of solver equations, however, involves far less data then the underlying visibility data.   

Even though the processing of the data will be done on a large cluster of computers, the total amount of data can be such that we expect the quality of the final result to be processing limited. This means that for all the algorithms we have to weight accuracy against the amount of Flops needed. It also means that the LOFAR instrument can be improved by upgrading the processing cluster in the future. 

\subsection{Processing steps}

LOFAR calibration is a joint estimation problem for both instrumental parameters and source parameters. At its heart lies the ``Measurement Equation'' that is used to model the observed data~\cite{Hamaker:96}. A signal processing data model and a Cramer-Rao lower bound analysis are given in ~\cite{Tol:07}. The latter paper also provides a good introduction to the signal processing aspects of LOFAR Self-Calibration.   

The final LOFAR calibration strategy is still under development. However, we foresee that the following steps and iterations will be part of it. The first step consists of removing bad data points, which are due to e.g. Radio Frequency Interference (RFI). After this step the contaminating contribution of a couple of very strong sources (like CasA, CygA, TauA, VirA) that enter through the station beam sidelobes needs to be removed. Since modelling the station beam sidelobes is infeasible due to the large number of parameters involved, the combined effect of the sources and the instrumental effects has to be estimated and subtracted from the data. 

Once the interfering signals are removed from the data, the data may be further integrated. The high resolution in frequency is only needed for removing RFI. The final resolution is determined by bandwidth smearing requirements~\cite{SIRAII:99}. In the frequency direction the data may be reduced by a factor of 3 to 10, depending on the size of the array used for the observation {\bf CHECK}. In principle the data may also be integrated along the time axis. Here, however, we have to make sure that the effect of the ionosphere remains constant over a time sample. The maximal reduction factor determined by time-average smearing ranges from 3 tot 10, again depending on array size {\bf CHECK}.

Next an iterative loop, dubbed the ``Major Cycle'', is entered where we first estimate instrumental and source parameters using the visibility data, then image the data, and finally refine the estimation  of the source parameters using image data. Since the estimation algoritms are also iterative in nature, this loop will be traversed a number of times~\cite{Nijboer:07}. 

After initial operation of the LOFAR instrument the parameters for the strongest sources will be known. From then on the strongest sources can be used in every observation to estimate ionospheric parameters, instrumental parameters, and to refine the estimate for the station beams that is available from the station calibration. Using these estimates the contributions for the strongest sources are removed from the data. The remaining residual visibility data is then imaged.

\ldots

Assume strong sources known. Refine beam parameters from station cal. Fit ionospheric parameters. \\
Subtract strong sources and fainter sources (with PSF far sidelobes above noise) \\
Correct, Integrate and Image in facets, using w-projection. Facet size determined by beam and ionosphere. \\
Update source parameters by source extraction. \\
Re-enter Major Cycle. Do on increasing subset of the data (time, freq, baseline). In total maximal ``twice trhough all the data''.



\ldots

The correlator produces visibility data on a 1 second and 0.62 kHz of 0.78 kHz resolution depending on the sampling frequency of the station processing. The frequency resolution is needed for excision of RFI signals. Strong RFI sources can in principle be suppressed on the station level by applying spatial filtering techniques~\cite{Veen1:04,Veen2:04,Boonstra:05}. However, applying these techniques on the station level would imply that all station beams will be different. That would mean that the number of station beam parameters to be estimated in the following self-calibration step would increase considerably. We foresee that this will be impractible in the initial LOFAR operations. However, in later upgrades these spatial filtering techniques may be incorporated. Initially, LOFAR processing will use traditional flagging techniques~\cite{Renting:07}.  

\subsection{Opmerkingen}

{\it focus on signal path}

{\it we will be processing limited}

\begin{itemize}
\item Flag RFI: papers Boonstra?
\item Take out the passband per subband
\item Compress the data, but avoid time and freq smearing
\item Major Cycle iteration
\item Fit Ionosphere, fit beam, Cat I Peeling: Papers Tol/Veen \& Tol/Jeffs/Veen
\item Cat II subtract
\item Facet Corrected Imaging
\item Shift the data per Facet
\item Integrate the data
\item Correct the data for the center of the Facet
\item Image the data using w-projection: Paper Cornwell
\item Extract sources from the image and update the source model
\item New iteration of Major Cycle, while refining the RFI flagging. Iterations will probably use increased levels of data resolution
\end{itemize}

\section{Current state and roll-out planning}

Currently, four partially-built stations are functional: 3~stations with
16~LBAs and 1~station with 48~LBAs.
To create more baselines and achieve better UV coverage, we usually split each
station into four {\em microstations}.
This yields 16~microstations, which are treated the same as real stations in
the online and offline processing pipelines.
A consequence of quadrupling the number of stations is that the bandwidth
is reduced to 36~subbands of 195~KHz or 48~subbands of 156~KHz.
Alternatively, the station with 48~LBA can be split into 12~microstations,
so that together with the other $3\times4$ microstations a total of
24~microstations can be formed,
but WAN restrictions limit the bandwidth to 12~resp.\ 16~subbands.
The HBA units are currently being commissioned.
A full LBA station in Effelsberg, Germany is also operational, and will soon
be connected via a dedicated wide-area link to the Central Processor.

Construction of the full stations is planned as follows.
\fixme{"station" should be consistent with HBA-core "double station"}
The first 18 full stations (13~core + 5~remote) \fixme{to be confirmed}
including the WAN links to the Central Processor will be operational by the
end of 2008, and the remaining 18~stations will be built in the course of 2009.
Meanwhile, construction of international stations will continue.
The Blue Gene/L is capable of handling all foreseen future data rates.

%\begin{figure*}
%\includegraphics[width=\textwidth]{LBA_observed.jpg}
%\caption{Real-time filters on the Central Processor.  For simplicity, the
%figure shows two stations, one subband, one polarization.}
%\label{fig:skymap}
%\end{figure*}

\section{Conclusion}
The conclusion goes here.





% if have a single appendix:
%\appendix[Proof of the Zonklar Equations]
% or
%\appendix  % for no appendix heading
% do not use \section anymore after \appendix, only \section*
% is possibly needed

% use appendices with more than one appendix
% then use \section to start each appendix
% you must declare a \section before using any
% \subsection or using \label (\appendices by itself
% starts a section numbered zero.)
%


%\appendices
%\section{Proof of the First Zonklar Equation}
%Appendix one text goes here.

% you can choose not to have a title for an appendix
% if you want by leaving the argument blank
\section{}
%Appendix two text goes here.


% use section* for acknowledgement
\section*{Acknowledgment}


The authors would like to thank the LOFAR team. ASTRON is a NWO institute.

LOFAR is funded by the Dutch government in the BSIK programme for
interdisciplinary research for improvements of the knowledge infrastructure.
Additional funding is provided by the European Union, European Regional
Development Fund (EFRO) and by the ``Samenwerkingsverband Noord-Nederland,''
EZ/KOMPAS.

% Can use something like this to put references on a page
% by themselves when using endfloat and the captionsoff option.
\ifCLASSOPTIONcaptionsoff
  \newpage
\fi



% trigger a \newpage just before the given reference
% number - used to balance the columns on the last page
% adjust value as needed - may need to be readjusted if
% the document is modified later
%\IEEEtriggeratref{8}
% The "triggered" command can be changed if desired:
%\IEEEtriggercmd{\enlargethispage{-5in}}

% references section

% can use a bibliography generated by BibTeX as a .bbl file
% BibTeX documentation can be easily obtained at:
% http://www.ctan.org/tex-archive/biblio/bibtex/contrib/doc/
% The IEEEtran BibTeX style support page is at:
% http://www.michaelshell.org/tex/ieeetran/bibtex/
%\bibliographystyle{IEEEtran}
% argument is your BibTeX string definitions and bibliography database(s)
%\bibliography{IEEEabrv,../bib/paper}
%
% <OR> manually copy in the resultant .bbl file
% set second argument of \begin to the number of references
% (used to reserve space for the reference number labels box)
%\begin{thebibliography}{1}
%
%\bibitem{IEEEhowto:kopka}
%H.~Kopka and P.~W. Daly, \emph{A Guide to \LaTeX}, 3rd~ed.\hskip 1em plus
%  0.5em minus 0.4em\relax Harlow, England: Addison-Wesley, 1999.
%
%\end{thebibliography}
\bibliographystyle{IEEEtran}
\bibliography{lofar,lofarRJN}

% biography section
% 
% If you have an EPS/PDF photo (graphicx package needed) extra braces are
% needed around the contents of the optional argument to biography to prevent
% the LaTeX parser from getting confused when it sees the complicated
% \includegraphics command within an optional argument. (You could create
% your own custom macro containing the \includegraphics command to make things
% simpler here.)
%\begin{biography}[{\includegraphics[width=1in,height=1.25in,clip,keepaspectratio]{mshell}}]{Michael Shell}
% or if you just want to reserve a space for a photo:

\begin{IEEEbiography}{Michael Shell}
Biography text here.
\end{IEEEbiography}

% if you will not have a photo at all:
\begin{IEEEbiographynophoto}{John Doe}
Biography text here.
\end{IEEEbiographynophoto}

% insert where needed to balance the two columns on the last page with
% biographies
%\newpage

\begin{IEEEbiographynophoto}{Jane Doe}
Biography text here.
\end{IEEEbiographynophoto}

% You can push biographies down or up by placing
% a \vfill before or after them. The appropriate
% use of \vfill depends on what kind of text is
% on the last page and whether or not the columns
% are being equalized.

%\vfill

% Can be used to pull up biographies so that the bottom of the last one
% is flush with the other column.
%\enlargethispage{-5in}



% that's all folks
\end{document}


