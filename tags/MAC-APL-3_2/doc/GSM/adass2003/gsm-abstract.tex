\documentclass[12pt]{article}

% \oddsidemargin=-5mm
% \oddsidemargin=-5mm
% \textwidth=170mm

\title{The LOFAR Global Sky Model: Design Challenges}

\begin{document}

\maketitle

\begin{abstract}

The LOFAR Global Sky Model (GSM) will be an all-sky database of some 100
million objects, with flux \& polarization measurements in the 20--200 MHz
range. The primary function of the GSM is to support LOFAR calibration and data
reduction. The GSM is expected to provide a model of all sufficiently  bright
sources in any given field, having enough detail and precision to calibrate and
subtract these sources and yield residual images of the faint background. The
GSM is expected to be continuously updated and refined during LOFAR operation
in a ``closed loop'' of sorts. The GSM is also a valuable stand-alone data
product that can be made compatible with the VO. 

The instrumental characteristics of LOFAR pose large challenges to GSM design,
some of them unique even in the field of very large catalogues. Besides sheer
size, this includes highly complex source models (thus making for a very
non-uniform database structure), stringent performance requirements for
operational use, the need to update source models operationally, and various
data management issues. This paper will focus on some of these challenges,
discuss our approaches to dealing with them, and present a prototype GSM being
developed for the LOFAR Pilot Selfcal System (PSS).

\end{abstract}

\end{document}

