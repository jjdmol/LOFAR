\documentclass{article}

\usepackage{spconf}

\title{How to Build a Correlator on Many-Core Hardware}

\name{Rob V. van Nieuwpoort and John W. Romein}
%\texttt{\{nieuwpoort,romein\}@astron.nl}}

\address{Stichting ASTRON (Netherlands Institute for Radio Astronomy) \\
Oude Hoogeveensedijk 4 \\
7991 PD\ \ Dwingeloo \\
The Netherlands}


\begin{document}

\maketitle

\begin{abstract}
\end{abstract}

\section{Introduction}

Radio telescopes produce enormous amounts of data.
The Low Frequency Array (LOFAR) stations~\cite{Butcher:04,deVos:09}, for
instance, will produce some tens of petabits per day; the dishes from the
Australian SKA Pathfinder (ASKAP) will even produce over six exabits per day.
To extract the sky signal from the system noise, the \emph{correlator\/}
correlates the signals by multiplying the samples of each pair of receivers.
Additionally, the correlator integrates correlations over time, to reduce
the amount of data.

Typically, custom-built hardware is used to correlate the signals.
A recent development is to use a supercomputer~\cite{Romein:06,Romein:09b}.
Both approaches have important advantages and disadvantages.
Custom-built hardware is efficient and consumes modest amounts of power, but is
inflexible, expensive to design, and has a long development time.
Solutions that use a supercomputer are much more flexible, but are less
efficient, consume more power, and are expensive to purchase.
Future instruments, like the Square Kilometre Array (SKA), need several orders
of magnitude more computational resources.
It is likely that the requirements of the SKA cannot be met by using
current supercomputer technology.

The unparalleled growth in the computational performance of Graphics Processor
Units (GPU) has attracted the attention of the high-performance computing 
community during the past five years.

%We do this using general-purpose multi-core processors, the Cell/B.E., and graphics processors
%(GPUs) from different vendors (ATI and NVIDIA).  For comparison
%reasons, we discuss the LOFAR production implementation on an IBM Blue
%Gene/P supercomputer.  There are many advantages to the use of
%many-cores: it is a flexible software solution, has lower costs in
%terms of purchase, is cheaper and easier to maintain, and the power
%usage per flop is significantly lower.  We use the LOFAR telescope as
%an illustrative example, but the results can be applied equally well on other
%instruments, including e-VLBI.




\bibliographystyle{IEEEbib}
\bibliography{spm}

\end{document}
