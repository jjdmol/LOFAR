\documentclass[]{lofar}
%
\usepackage{layout}
%
%
% Define the information of the page header
%
\renewcommand{\theAuthor}  {D.\ Hoogland}
\renewcommand{\theDOIssue} {2003-Jul-02}
\renewcommand{\theKOIssue} {Public}
\renewcommand{\theStatus}  {Draft}
\renewcommand{\theRevNR}   {0.1.1}
\renewcommand{\theScope}   {CEP}
\renewcommand{\theDocNR}   {LOFAR-ASTRON-DOC-xxx}
\renewcommand{\theFile}    {ftp://kant.astron.nl/pub/tex/lofar-cls-0.9.6.tgz}
%
% Define the title of the document
%

\renewcommand{\theTitle}
{Blackboard, a framework for imaging applications}

\renewcommand{\theTAuthor} {D.\ Hoogland}
%
% Define the names in the Verified table
%
\renewcommand{\theVNameList} {K.\ van der Schaaf\\
                              K.\ van der Schaaf}
\renewcommand{\theVSigList}  {o.p.v. \\
                              ......................................
                             }
\renewcommand{\theVDateList} {2003-Jul-02\\
                              .....................
                             }
\renewcommand{\theVRevList}  {0.1.0 \\
                              0.1.1 }
%
% Define the names in the Accepted table
%
\renewcommand{\theAHeada}    {Work Package Manager}
\renewcommand{\theAHeadb}    {System Engineering Manager}
\renewcommand{\theAHeadc}    {Program Manager}
\renewcommand{\theAManagera} {D.\ Hoogland}
\renewcommand{\theAManagerb} {K.\ van\ der\ Schaaf}
\renewcommand{\theAManagerc} {C.M.\ de\ Vos}
%
% This goes into the footer
%
\renewcommand{\theProject}   {LOFAR Project}
%
% Define the first,second and third columns of the distribution list
%
\renewcommand{\dGList} {%
                        ASTRON\\
                        \hspace{0.5cm} D.\ Hoogland\\
                        \hspace{0.5cm} K.\ van\ der\ Schaaf
                        \hspace{0.5cm} G.\  van Diepen\\
                        \hspace{0.5cm} A.\  van Gerdes\\
                        \hspace{0.5cm} J.\ Noordam
                       }
%
\renewcommand{\ddGList}{%
                        \\
                        J.\ Reitsma\\
                        O.\ Smirnov\\
                        C.M.\ de Vos
                       }
%
\renewcommand{\dOList}{%
                        ORDINA\\
                        \hspace{0.5cm} K-J.\ Wieringa\\[\baselineskip]
                        Snow B.V.\\
                        \hspace{0.5cm} D.\ Hoogland
}
%
% Define the version history
%
\renewcommand{\revList}     {%
                             0.0.1\\
                             0.1.0\\
                             0.1.1
                            }
\renewcommand{\dateList}    {%
                             2003-Jun-18\\
                             2003-Jun-24\\
                             2003-Jul-02
                            }
\renewcommand{\sectionList} {%
                             -\\
                             \ref{Sec2}\\
                             -
                            }
\renewcommand{\pageList}    {%
                             -\\
                             \pageref{Sec2}-\pageref{Sec2e}\\
                             -
                            }

\renewcommand{\changeList} {%
                            Creation in xml\\
                            rough sketch\\
                            Creation \LATEX version
                           }

%
\begin{document}
\maketitle
\newpage
%
\theDistributionList
\vspace{1cm}
\theDocumentRevision
\newpage
%
\begin{abstract}
The author has written a document class file \textsf{lofar.cls}, which
may be used to type set reports with \LaTeX. The document class takes
care of type setting a report according to the LOFAR documentation
standard. In this note a description of this document class is
presented. This document is only useful for people who write there
reports with \LaTeX.
\end{abstract}
\newpage
%
\tableofcontents
\newpage

\section{Introduction}
\label{Sec1}

\PARstart{T}{his} document describes the document class \textsf{lofar.cls}, 
which can be used for type setting a LOFAR report with \LaTeX. If this
document class file is used, the report will be layed out according to
the LOFAR project standard.  In the remainder of this document the
features of this class file will be discussed. It is assumed that the
reader of this document is familiar with \LaTeX\ concepts.

\section{Description of the document class}
\label{Sec2}
%
The typewriter face will be used for quoting text of the \LaTeX\
document source.  The features of the lofar document class are
described using the source code of this \LaTeX\ as a reference.

\subsection{The document header}

The document header is the first section of the \LaTeX\ source up to
the line with \verb=\begin{document}=.
%
\begin{verbatim}
\documentclass[]{lofar}
%
\usepackage{amsmath}
%
%
% Define the information of the page header
%
\renewcommand{\theAuthor}  {D.\ Hoogland}
\renewcommand{\theDOIssue} {2003-Jul-02}
\renewcommand{\theKOIssue} {Public}
\renewcommand{\theStatus}  {Draft}
\renewcommand{\theRevNR}   {0.1.1}
\renewcommand{\theScope}   {CEP}
\renewcommand{\theDocNR}   {LOFAR-ASTRON-DOC-xxx}
\renewcommand{\theFile}    {ftp://kant.astron.nl/pub/tex/lofar-cls-0.9.6.tgz}
%
% Define the title of the document
%

\renewcommand{\theTitle}
{Blackboard, a framework for imaging applications}

\renewcommand{\theTAuthor} {D.\ Hoogland}
%
% Define the names in the Verified table
%
\renewcommand{\theVNameList} {K.\ van der Schaaf\\
                              K.\ van der Schaaf}
\renewcommand{\theVSigList}  {o.p.v. \\
                              ......................................
                             }
\renewcommand{\theVDateList} {2003-Jul-02\\
                              .....................
                             }
\renewcommand{\theVRevList}  {0.1.0 \\
                              0.1.1 }
%
% Define the names in the Accepted table
%
\renewcommand{\theAHeada}    {Work Package Manager}
\renewcommand{\theAHeadb}    {System Engineering Manager}
\renewcommand{\theAHeadc}    {Program Manager}
\renewcommand{\theAManagera} {D.\ Hoogland}
\renewcommand{\theAManagerb} {K.\ van\ der\ Schaaf}
\renewcommand{\theAManagerc} {C.M.\ de\ Vos}
%
% This goes into the footer
%
\renewcommand{\theProject}   {LOFAR Project}
%
% Define the first,second and third columns of the distribution list
%
\renewcommand{\dGList} {%
                        ASTRON\\
                        \hspace{0.5cm} D.\ Hoogland\\
                        \hspace{0.5cm} K.\ van\ der\ Schaaf
                        \hspace{0.5cm} G.\  van Diepen\\
                        \hspace{0.5cm} A.\  van Gerdes\\
                        \hspace{0.5cm} J.\ Noordam
                       }
%
\renewcommand{\ddGList}{%
                        \\
                        J.\ Reitsma\\
                        O.\ Smirnov\\
                        C.M.\ de Vos
                       }
%
\renewcommand{\dOList}{%
                        ORDINA\\
                        \hspace{0.5cm} K-J.\ Wieringa\\[\baselineskip]
                        Snow B.V.\\
                        \hspace{0.5cm} D.\ Hoogland
}
%
% Define the version history
%
\renewcommand{\revList}     {%
                             0.0.1\\
                             0.1.0\\
                             0.1.1
                            }
\renewcommand{\dateList}    {%
                             2003-Jun-18\\
                             2003-Jun-24\\
                             2003-Jul-02
                            }
\renewcommand{\sectionList} {%
                             -\\
                             \ref{Sec2}\\
                             -
                            }
\renewcommand{\pageList}    {%
                             -\\
                             \pageref{Sec2}-\pageref{Sec2e}\\
                             -
                            }

\renewcommand{\changeList} {%
                            Creation in xml\\
                            rough sketch\\
                            Creation \LATEX version
                           }

%
\begin{document}
\end{verbatim}
%
\LaTeX2e\ is assumed, as can be concluded from the first line. The class 
file has not been tested with the previous version of \LaTeX. The
lofar document class is selected by choosing lofar as the class option
of \verb=\documentclass=. It is obvious that \LaTeX\ should be able to
find the file \textsf{lofar.cls}.
%
The lofar document class takes care of type setting the front page,
the page headers and footers and much more. As usual, in the document
header, packages can be selected with \verb=\usepackage= statemens. In
this example, all document meta information is gathered into a
seperate file called \textsf{definitions.tex}. The data are included
in the document header by means of the \verb=%
% Define the information of the page header
%
\renewcommand{\theAuthor}  {D.\ Hoogland}
\renewcommand{\theDOIssue} {2003-Jul-02}
\renewcommand{\theKOIssue} {Public}
\renewcommand{\theStatus}  {Draft}
\renewcommand{\theRevNR}   {0.1.1}
\renewcommand{\theScope}   {CEP}
\renewcommand{\theDocNR}   {LOFAR-ASTRON-DOC-xxx}
\renewcommand{\theFile}    {ftp://kant.astron.nl/pub/tex/lofar-cls-0.9.6.tgz}
%
% Define the title of the document
%

\renewcommand{\theTitle}
{Blackboard, a framework for imaging applications}

\renewcommand{\theTAuthor} {D.\ Hoogland}
%
% Define the names in the Verified table
%
\renewcommand{\theVNameList} {K.\ van der Schaaf\\
                              K.\ van der Schaaf}
\renewcommand{\theVSigList}  {o.p.v. \\
                              ......................................
                             }
\renewcommand{\theVDateList} {2003-Jul-02\\
                              .....................
                             }
\renewcommand{\theVRevList}  {0.1.0 \\
                              0.1.1 }
%
% Define the names in the Accepted table
%
\renewcommand{\theAHeada}    {Work Package Manager}
\renewcommand{\theAHeadb}    {System Engineering Manager}
\renewcommand{\theAHeadc}    {Program Manager}
\renewcommand{\theAManagera} {D.\ Hoogland}
\renewcommand{\theAManagerb} {K.\ van\ der\ Schaaf}
\renewcommand{\theAManagerc} {C.M.\ de\ Vos}
%
% This goes into the footer
%
\renewcommand{\theProject}   {LOFAR Project}
%
% Define the first,second and third columns of the distribution list
%
\renewcommand{\dGList} {%
                        ASTRON\\
                        \hspace{0.5cm} D.\ Hoogland\\
                        \hspace{0.5cm} K.\ van\ der\ Schaaf
                        \hspace{0.5cm} G.\  van Diepen\\
                        \hspace{0.5cm} A.\  van Gerdes\\
                        \hspace{0.5cm} J.\ Noordam
                       }
%
\renewcommand{\ddGList}{%
                        \\
                        J.\ Reitsma\\
                        O.\ Smirnov\\
                        C.M.\ de Vos
                       }
%
\renewcommand{\dOList}{%
                        ORDINA\\
                        \hspace{0.5cm} K-J.\ Wieringa\\[\baselineskip]
                        Snow B.V.\\
                        \hspace{0.5cm} D.\ Hoogland
}
%
% Define the version history
%
\renewcommand{\revList}     {%
                             0.0.1\\
                             0.1.0\\
                             0.1.1
                            }
\renewcommand{\dateList}    {%
                             2003-Jun-18\\
                             2003-Jun-24\\
                             2003-Jul-02
                            }
\renewcommand{\sectionList} {%
                             -\\
                             \ref{Sec2}\\
                             -
                            }
\renewcommand{\pageList}    {%
                             -\\
                             \pageref{Sec2}-\pageref{Sec2e}\\
                             -
                            }

\renewcommand{\changeList} {%
                            Creation in xml\\
                            rough sketch\\
                            Creation \LATEX version
                           }
=
statement.  The information which can be defined is:
%
\begin{enumerate}
\item The information of the page header:
        \begin{enumerate}
           \item The author of the document;
           \item The date of issue;
           \item The kind of issue;
           \item The document status;
           \item The document revision number;
           \item The document scope;
           \item A unique document number.
        \end{enumerate}
\item The title of the document;
\item Lists with names an revision numbers to be placed in the
  ``Verified'' table on the front page;
\item The descriptions and names of the managers to be placed in the
  ``Accepted'' table on the frontpage;
\item A string (here defined as ``LOFAR Project'') to be placed in the 
      page footer;
\item The names to be placed in the distribution list;
\item The information to be placed in the table defining the document 
      revision history.
\end{enumerate} 
%
It is not compulsory to divert this data into a seperate file as has
been done for this document. As an alternative, the document meta data
can be defined directly in the document header section. For each piece
of meta data, an empty \LaTeX\ command is defined in the document
class. The actual information is assigned to the corresponding
commands by means of \verb=\renewcommand= statements specified by the
class user. The commands are used as macro's. They expand to the
assigned macro values when the document is processed by \LaTeX.

\subsection{Type setting the front page}
%
\begin{verbatim}
%
\begin{document}
\maketitle
\newpage
%
\end{verbatim}
%
The command \verb=\maketitle= takes care of type setting the first two
pages. Also the frontpage contains the page header. The information in
the page header is defined by the following commands in the file
\textsf{defenitions.tex}:
\newpage
%
\begin{verbatim}
%
% Define the information of the page header
%
\renewcommand{\theAuthor}  {G.W.\ Kant}
\renewcommand{\theDOIssue} {2002-Feb-27}
\renewcommand{\theKOIssue} {Public}
\renewcommand{\theStatus}  {Draft}
\renewcommand{\theRevNR}   {0.9.5}
\renewcommand{\theScope}   {Project Documentation}
\renewcommand{\theDocNR}   {LOFAR-ASTRON-DOC-005}
\end{verbatim}
%
The document title and author to be displayed below the title are defined by
%
\begin{verbatim}
%
% Define the title of the document
%
\renewcommand{\theTitle}   { Publishing LOFAR reports with the LOFAR \LaTeX\ 
                             document class named lofar.cls}
\renewcommand{\theTAuthor} {G.W.\ Kant}
\end{verbatim}
%
and the rest of the frontpage information is defined by
%
\begin{verbatim}
%
% Define the info in the Verified table
%
\renewcommand{\theVNameList} {K.\ van der Schaaf\\
                              K.\ van der Schaaf}
\renewcommand{\theVSigList}  {o.p.v. \\
                             }
\renewcommand{\theVDateList} {2002-Feb-22\\
                             }
\renewcommand{\theVRevList}  {0.9.1 \\
                              0.9.5 }
%
% Define the names in the Accepted table
%
\renewcommand{\theAHeada}    {Work Package Manager}
\renewcommand{\theAHeadb}    {System Engineering Manager}
\renewcommand{\theAHeadc}    {Program Manager}
\renewcommand{\theAManagera} {C.M.\ de Vos}
\renewcommand{\theAManagerb} {C.M.\ de Vos}
\renewcommand{\theAManagerc} {J.\ Reitsma}
%
% This goes into the footer
%
\renewcommand{\theProject}   {LOFAR Project}
\end{verbatim}
%

\subsection{The distribution list}

The distribution list is generated with the command 
\verb=\theDistributionList= following the \verb=\newpage= after the
\verb=\maketitle= command.
%
\begin{verbatim}
%
\begin{document}
\maketitle
\newpage
%
\theDistributionList
%
\end{verbatim}
%
The names which should go into the table are specified by the following 
definitions in \textsf{definitions.tex}.
%
\begin{verbatim}
%
% Define the first and second columns of the distribution list
%
\renewcommand{\dGList} {%
                        ASTRON\\
                        \hspace{0.5cm} M.\ Arts\\
                        \hspace{0.5cm} A.J.\ Boonstra\\
                        \hspace{0.5cm} J.D.\ Bregman\\
                        \hspace{0.5cm} G.W.\ Kant\\
                        \hspace{0.5cm} J.\ Morawietz\\
                        \hspace{0.5cm} J.\ Noordam\\
                        \hspace{0.5cm} J.\ Reitsma\\
                        \hspace{0.5cm} C.M.\ de Vos
                       }
\renewcommand{\ddGList}{%
                       }
%
\renewcommand{\dOList}{%
                        ORDINA\\
                        \hspace{0.5cm} E.\ Lauwerman\\
                        \hspace{0.5cm} F.\ van Eck\\[\baselineskip]
                        Fokker Space\\
                        \hspace{0.5cm} J.\ Doornink\\
                        \hspace{0.5cm} J.H.\ Kollen\\
                        \hspace{0.5cm} E.\ Overbosch\\
                        \hspace{0.5cm} H.\ de Wolf
}
%
\end{verbatim}
%
Two columns of names can be specified in the Group column. Here, only
one column is specified with the \verb=\dGList=.

\subsection{The document revision}

The document revision table is generated with the command 
\verb=\theDocumentRevision=. Here, 1~cm additional vertical space 
is added between the ``Distribution List'' and the ``Document
Revision'' section.
%
\begin{verbatim}
%
\begin{document}
\maketitle
\newpage
%
\theDistributionList
\vspace{1cm}
\theDocumentRevision
\newpage
%
\end{verbatim}
%
The revision information which should go into the table are specified
by the following definitions in \textsf{definitions.tex}. Notice the
use of the \verb=\theRevNR= command in the
\verb=\revList=. Furthermore the use of conventional \LaTeX\ reference
operators is demonstrated.
%
\begin{verbatim}
%
% Define the version history
%
\renewcommand{\revList}     {%
                             0.5\\
                             \theRevNR
                            }
\renewcommand{\dateList}    {%
                             2002-Feb-20\\
                             \theDOIssue
                            }
\renewcommand{\sectionList} {%
                             -\\
                             \ref{Sec2}
                            }
\renewcommand{\pageList}    {%
                             -\\
                             \pageref{Sec2}-\pageref{Sec2e}
                            }

\renewcommand{\changeList} {%
                            Creation\\
                            Updated class description
                           }
\end{verbatim}

\label{Sec2e}

%\section{Page geometry}
%
%\layout
%\label{Sec3e}

\section{Conclusion}
\label{Sec4}
A \LaTeX\ document class has been written, which may be used to type
set LOFAR reports according to the specified LOFAR document standard.
In this note a description is given of most of the features of this
document class.
\end{document}
