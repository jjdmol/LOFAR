\documentclass[a4paper]{article}

\title{CITT Release Notes}

\begin{document}
\maketitle

Release Notes - awimager cittg 0.1

The major change in this release is a re-factored code base.
That means that a lot of the code has changed, but the functionality has remained mostly the same.
The purpose of these changes is to make the code more readable and modular. This allows for faster development
of new features. It also makes it easier for new developers to contribute to the imager.

Not all of the existing functionality has been ported to the new framework yet. So in some respects the functionality of
this release less then the current LOFAR trunk version (LofIm and Pipeline).

New functionality in this release was introduced by and upgrade of the underlying casa libraries
from casarest ( = casa v2 from sept 2009) to Casa 4.2.

Improvements in multiscale and wideband imaging


Changes to the interface

Parameters have been renamed and grouped together using the dot notation used in parsets, e.g. 
npix has become image.npix, padding has become gridding.padding


Test program

The first goal of the test program is to verify that existing functionality has indeed been ported correctly to
the new framework.

Testing

* generation of test data
  - create a ms with makems
  - add the beam tables
  - add a WEIGHT\_SPECTRUM column
  - set flags in the beam tables 

The LOFAR beam are fairly symmetric. To make it easier to find errors it help to introduce an artificial phase
gradient by flagging half of the antennas at one side of a station.

Create an empty image

Set one or more pixels

Create BBS sky model

Run BBS

Run awimager with operation=predict

Compare CORRECTED\_DATA and MODEL\_DATA


TODO:

* User selection of WEIGHT or WEIGHT\_SPECTRUM column
* enable uniform and robust weighting
* enable split beam mode (potentially ~4 speedup)
* correct for artifical steeping of the the spectrum of sources at the edge of the FOV



\end{document}