\documentclass[a4paper,twoside, 10pt]{report}
\usepackage{graphicx}
\usepackage{natbib}
\usepackage{xspace}
\usepackage{listings}

%
%  P R E A M B L E
%

\title{Cobalt commissioning report}
\author{M.A. Brentjens}

\newcommand{\cobalt}{Cobalt\xspace}
\newcommand{\bgp}{BG/P\xspace}
\newcommand{\cep}{CEP-2\xspace}

\newcommand{\hms}[3]{\ensuremath{#1^\mathrm{h} #2^\mathrm{m} #3^\mathrm{s}}}
\newcommand{\dms}[3]{\ensuremath{#1^\circ #2' #3''}}


\newcounter{experimentc}
\newenvironment{experiment}[4][TODO]{\hspace{1ex}\linebreak
\noindent%\begin{minipage}{\columnwidth}%
\noindent\refstepcounter{experimentc}\textbf{Exp. \arabic{experimentc}
  [#1]:\ %
    #2  [#3]}\\%
\textbf{Aim: } \emph{#4}%
\addcontentsline{toc}{section}{Exp.~\arabic{experimentc} [{\footnotesize
    #1}]:~#2}%
\ignorespaces%\begin{center}%\begin{minipage}{0.9\columnwidth}%
\setlength{\parindent}{2em}}%
{%\end{minipage}
%\end{center}
\ \linebreak
\noindent\rule{0.1\textwidth}{0.4pt}
%\end{minipage}
\linebreak
\par\noindent\ignorespacesafterend}



%
%  D O C U M E N T
%

\begin{document}

\maketitle

\tableofcontents

\chapter{Introduction}

\cobalt is the correlator that was supposed to succeed the \bgp at
2013-12-31 at the latest \citep{CobaltRequirements2013}. Due to some
delays, it has actually replaced the \bgp at the end of April 2014. In
the scenario-1 design \citep{CobaltHardwareDesign2013}, \cobalt
comprises 8 PC nodes, equipped with 2 NVIDIA GPU units each,
interconnected by an infiniband switch.

This document describes in as much detail as possible, from a
technical-scientific point of view, \emph{what} must be tested,
\emph{how} it should be tested, and when and how it was tested. At the
end of commissioning, we want the correlator to

\begin{description}
\item[be feature complete:] it must be able to perform the same types of
  observations as the \bgp;
\item[be correct:] the \cobalt output must be the same as the \bgp output,
  within the expected numerical precision;
\item[perform well:] support all \bgp observation types with an
  execution speed the same or better than the \bgp;
\item[be robust:] the correlator should not crash on errors in the
  specification, and if it does, it should leave clear error messages.
\item[well integrated:] we must be able to simply specify and
  observation in MoM, schedule it, have it run on \cobalt, get data
  written to \cep, pipelines started, and obtain pipeline products
  that can be ingested into, and retrieved from the long term archive
  (LTA).
\end{description}

Because \cobalt is built from commodity hardware, and all interesting
functionality is implemented in software
\citep{CobaltSoftwareDesign2013} running on CPUs and GPUs, many
experiments described in this document can be conducted with
synthetic or pre-recorded data.

In this document, the word ``pipeline'' refers to specific \cobalt
code that processes incoming data during an observation, not post
processing software that usually runs after an observation has
completed, unless explicitly stated otherwise.

%% This document is subject to change and new suggestions. Please notify
%% me if I have forgotten to test important functionality, or if errors
%% or mistakes are found. New experiments and tests will ony be added if
%% they test specific \cobalt functionality that is not yet covered by
%% the current experiments.



%% \chapter{Test data sets}

%% \begin{table}
%% \caption{Summary of test data sets \cobalt. Both sets must be
%%   taken in 8-bit as well as 16-bit mode close to transit.}
%% \label{tab:test-data-sets}
%% \begin{tabular}{lp{12em}p{12em}}
%% \hline
%% \hline
%% Data set:     & 3C~295                 & PSR B0329+54\\
%% \hline
%% RA            & \hms {14}{11}{20.6} & \hms{03}{32}{59.37}\\
%% Dec           & \dms{+52}{12}{09.0}  & \dms{+54}{34}{44.9}\\
%% Stations      & all (minus CS013)      & core (minus CS013)\\
%% Antenna field & HBA\_DUAL              & HBA\_DUAL \\
%% Sub bands     &
%% 77,92,108,124,139,155,
%% 170,186,201,217,233,248,
%% 264,279,295,311,326,342,
%% 357,373,388,404,420,435        & 256..291\\
%% Duration      & 60 s                   & 60 s\\
%% Clock         & 200 MHz                & 200 MHz\\ 
%% Data volume   & $2\times 150$ GB       & $2\times 150$ GB\\
%% \hline
%% \hline
%% \end{tabular}
%% \end{table}


%% Many of the commissioning tests can be implemented as regression tests
%% if done with either pre-recorded UDP packets from the stations or
%% simulated time series.

%% The pre-recorded test data are stored on CEP2. One station producing
%% 244 sub bands of 16 bit data at a clock frequency of 200~MHz, produces
%% $3.05\cdot 10^9$ bits per second. For 70 antenna fields, this amounts
%% to 1.5~TB per minute. For 48 core fields, it is about 1~TB per minute.

%% I propose to record 4 test data sets: 8 bit and 16 bit data of 3C~295
%% to test the correlator, and 8 bit and 16 bit data of PSR~B0329+64 to
%% test the beamformed pipelines.  Table~\ref{tab:test-data-sets} lists
%% the specifications of the test data sets. The total data volume is
%% 600~GB. One can easily extract a much smaller sub set of data for
%% specific experiments.

%% It has been agreed with the \cobalt developers, that they will provide
%% a utility that can read time series data, and convert those to
%% simulated station series to feed to \cobalt. These synthetic data are
%% invaluable to directly test the mathematical operations that \cobalt
%% %% implements.



\chapter{Basic sanity}


\begin{experiment}[PASSED 2013-11-21]
{Reproducibility (correlator)}
{recorded data}
{Running the same input UDP streams through cobalt using the same
  parset must produce bit-for-bit the same output data.}

\noindent This is a very simple experiment. Just play back the same
observation twice and compare the output. After some initial issues
with the circular buffer code, this test passed on data set
L189429. We used the following function for validation.

{\small
\begin{lstlisting}[language=Python]
def are_they_identical(ms_file_name, ms_ref_file_name):
    tab     = table(ms_file_name).query('ANTENNA1 != ANTENNA2')
    tab_ref = table(ms_ref_file_name).query('ANTENNA1 != ANTENNA2')
    data     = tab.getcol('DATA')
    data_ref = tab_ref.getcol('DATA')
    uvw     = tab.getcol('UVW')
    uvw_ref = tab_ref.getcol('UVW')
    if len(uvw) != len(uvw_ref) or abs((uvw - uvw_ref)).max() != 0.0:
        raise ValueError('Uvw coordinates differ.')
    uv_distance = sqrt(uvw[:,0]**2 + uvw[:,1]**2)
    diff        = data - data_ref
    if abs(diff).max() != 0.0:
        raise ValueError('Data sets are not identical')
    return True
\end{lstlisting}
}
\end{experiment}




\begin{experiment}[PASSED 2014-03-10]
{Reproducibility (beam former)}
{recorded data}
{Running the same input UDP streams through cobalt using the same
  parset must produce bit-for-bit the same output data.}

\noindent Repeat the previous experiment for the beam formed modes. Do this for
both coherent stokes, incoherent stokes, with, and without coherent
dedispersion.

This test has been incorporated in the automated test suite, and runs
daily. Coherent dedispersion is not yet included.
\end{experiment}







\chapter{Channels}

The channelization of the input sub bands is the first stage in both
the correlator pipeline and the beam former pipeline. 


\begin{figure}
\begin{center}
\includegraphics[width=0.7\columnwidth]{correlator-channel-bandpass-80-81-20130605.png}
\end{center}
\caption{Measurement of channel band pass and channel separation by
  injecting a sinusoidal signal at various frequencies.}
\label{fig:channel-bandpass-corr}
\end{figure}


\begin{experiment}[PASSED 2013-06-05]
{Channel isolation}
{synthetic data}
{Measure the band pass of individual correlator output channels.}

\noindent Generate synthetic CW input signals at a prime number of
frequencies, beginning at the centre of one channel, and increasing
the frequency until at the centre of a neighbouring channel. Use
e.g. 44 frequencies, where the first and last should be precisely at
channel centres. This leaves 43 intervals and avoids resonances with
other constants in the code base. Plot the real part of the output as
a function of frequency.


This experiment has been conducted on 2013-06-05 by Wouter Kleijn. The
bandpass is plotted in Fig.\ref{fig:channel-bandpass-corr}.
\end{experiment}










\begin{experiment}[PASSED 2013-11-21]
{Frequency labelling: coarse}
{recorded data}
{Determine the correctness of frequency labelling in the output MS.}

\noindent There are two strong transmitters in the HBA\_LOW band at sub band 357
that can help verify the labelling of channel frequencies, as well as
the order in which channels are written. A channel's frequency label
represents the channel's \emph{centre}. The transmitters to look for
are
\begin{itemize}
  \item P2000 at 169.650 MHz;
  \item KPN ERMES at 169.750 MHz.
\end{itemize}

The central frequency (middle of channel 128, counting from 0) of sub
band 357 is 169\,726\,562.5 Hz. When operating at 256 channels per sub
band, the channel width is 762.939453 Hz. The P2000 peak should
therefore show up in channel 28 (counting from 0). In fact, the peak
must be 0.148 channel widths, or 113 Hz, from the lower boundary of
channel 28. KPN ERMES, if within reach, should peak in channel 159,
about 0.22 channel widths, or 168 Hz, from the lower boundary of the
channel, but the exact position within a channel is not easy to
extract from this test because the carriers are so narrow.

For this experiment, pre-recorded station data of this sub band should
be fed to \cobalt. The auto-correlated visibilities must be averaged
to 1 second, and the MS stored. Only 10 seconds of data from one
station are required.

Considering that the beam forming pipeline and the cross correlating
pipelines have different approaches to split a sub band into channels,
this should be conducted for \emph{both} pipelines. Ideally, this
experiment is conducted for 200~MHz as well 160~MHz data. 


This sub band ends up being SB018 in the test data sets. This was
tested with data set L189429 (16 bit, 3C~295) on 2013-11-20. The
associated plot is shown in
Fig.~\ref{fig:frequencies-of-P2000-and-ERMES}.

We did not yet conduct the 160~MHz experiment.
\end{experiment}

\begin{figure}
\begin{center}
\includegraphics[width=\columnwidth]{L189429-SB018-spectrum.pdf}
\end{center}
\caption{Spectrum of sub band 357 containing the ERMES and P2000
 signals for data set L189249. This is the average spectrum over all
 baselines, both cross correlations and auto correlations. The
 vertical black lines are at the listed carrier frequencies for these
 transmitters.}
\label{fig:frequencies-of-P2000-and-ERMES}
\end{figure}





\begin{figure}
\begin{center}
\includegraphics[width=\columnwidth]{cobalt-detailed-frequency-verification.pdf}
\end{center}
\caption{Spectral response of the 64 bit float precision time series
  (red) and the autocorrelation spectrum of the corresponding 8-bit
  integer time series, as processed by \cobalt without correcting for
  the station sub band bandpass. Channel width is about 3~kHz, so
  frequency labelling is approximately good to 1\% of the channel
  width.}
\label{fig:detailed-frequency-labels}
\end{figure}


\begin{experiment}[PASSED 2014-01-21]
{Frequency labelling: fine}
{synthetic data}
{Determine correctness of frequency labelling within a channel}

\noindent Prepare a timeseries by taking a spectrum with a gaussian
amplitude as a function of frequency, where the peak of the gaussian
is about 1/3 of a channel away from the channel edge with a $\sigma$
of about 2 channels, and Fourier transforming this spectrum to a time
series at a time interval of 5.12~$\mu$s.

Feed the time series to \cobalt, and look at the generated sub
band. Fit a gaussian to the channel amplitudes and determine its
peak. It should correspond to the exact frequency that was originally
put in.

We have done this experiment slightly differently, generating the time
series from a complex Gaussian noise time series, FIR filtered with a
filter response in the frequency domain that had a 12 kHz standard
deviation. The exact procedure is described in ipython notebook
frequency-scale-cobalt.ipynb, which is available from M.A. Brentjens
upon request.

The results are shown in Fig.~\ref{fig:detailed-frequency-labels}. We
have verified that the labelled central frequency of the channels is
good to about 1\% of the channel width. This is good enough for all
science cases we are aware of. The current experiment gives no reason
to believe that the frequency labelling is less than perfect.
\end{experiment}







\begin{figure}
\includegraphics[width=\columnwidth]{L189598-Cyg-A-station-bandpass-correction-16-64-256.pdf}
\includegraphics[width=\columnwidth]{L189598-Cyg-A-station-bandpass-correction-16-64-256-zoom.pdf}
\caption{Station sub band band pass correction on / off (correlator). Note
  the scale in the bottom panel.}
\label{fig:station-subband-bandpass-corr}
\end{figure}


\begin{figure}
\includegraphics[width=\columnwidth]{cs-station-bandpass-correction}
\caption{Station sub band band pass correction on / off (beamformer
  coherent stokes). Note the scale in the bottom panel.}
\label{fig:station-subband-bandpass-cs}
\end{figure}



\begin{figure}
\includegraphics[width=\columnwidth]{is-station-bandpass-correction}
\caption{Station sub band band pass correction on / off (beamformer
  incoherent stokes). Note the scale in the bottom panel.}
\label{fig:station-subband-bandpass-is}
\end{figure}


\begin{experiment}[PASSED (corr) 2013-11-22 / INSUFFICIENT
    (beamformer) 2014-08-28]
{Station band pass correction}
{recorded data}
{Verify that station band pass correction works.}

\noindent The stations use a poly phase filter (PPF) to separate the antenna's
time series into 512 sub bands. This leaves a distinctive wobbly
pattern in a sub band's band pass. \cobalt must be able to correct
this. 

Pre-record 24 adjacent HBA sub bands in an area of spectrum that is
relatively free of interference. 10 seconds of data would suffice. Use
two core stations and two remote stations. Produce plots of the
amplitude and phase of the concatenated ten sub band spectrum for all
combinations of settings in the following grid:


\begin{center}
\begin{tabular}{l|l|l|l}
\hline
\hline
 Correction        & 16 ch/sb& 64 ch/sb& 256 ch/sb\\
\hline
on  (cross-corr)   & DONE   & DONE   &  DONE   \\
off (cross-corr)   & DONE   & DONE   &  DONE   \\
\hline
on  (beam formed)  & DONE   & DONE   & DONE    \\
off (beam formed)  & DONE   & DONE   & DONE    \\
\hline
\hline
\end{tabular}
\end{center}


In reality, we have averaged all baselines, and in the beam former
case, all core stations. Plotting the phase was therefore deemed not
very useful, and meaningless in the beam former case, because the
output spectra there are real valued anyway.

We used sub bands 113\ldots136 (122.07~--126.56~MHz central
frequencies), which are generally very clean. in terms of RFI. In
fact, we used all Dutch stations except for CS013, CS401, and RS210 in
HBA\_DUAL\_INNER mode. The observation lasted for 16 seconds. The
first and last three seconds were discarded to avoid potential
start/end effects. We observed Cyg A near
transit. Figure~\ref{fig:station-subband-bandpass-corr} shows the
results for the correlator. It is clear that the correction works the
same way it did on the \bgp. The residual slope is due to the mean
antenna and receiver band pass in the lower part of the HBA band.

For the beamformer (Figs.~\ref{fig:station-subband-bandpass-cs}
and~\ref{fig:station-subband-bandpass-is}) the correction is
reasonable, but there is some residual from the station band
pass. Additionally, there appears to be small ripple as a function of
frequency. These things needs to be investigated further during the
regular operational phase for \cobalt. Given that the residual shape
is bigger in the IS case, I suspect that it has something to do with
the station's autocorrelations, which are not included in the
correlator plot. They are included (along with all cross correlations)
in the CS plot, and exclusively make up the IS plot.
\end{experiment}










\chapter{Delay compensation}


\begin{experiment}[PASSED 2013-11-15]
{Order of antennas and sign of baseline vector}
{recorded data}
{Verify that ANTENNA1$\le$ANTENNA2 and that UVW = ANTENNA2 - ANTENNA1}

\noindent Run the test case \texttt{test\_ant\_columns} from the
program \texttt{verify-ms-format.py} listed in appendix
\ref{sec:verify-ms-format}. This was done for L188487, which passed
the test case on 2013-11-15.

\end{experiment}




\begin{figure}
\begin{center}
\includegraphics[width=0.8\textwidth]{L189429-1x1-single-vs-double-sincos.pdf}
\end{center}
\caption{Effect of single precision in sin cos calculations for
  fringe stopping. Histogram of differences and ratios.}
\label{fig:single-vs-double-diff-ratio-histograms}
\end{figure}

\begin{figure}
\begin{center}
\includegraphics[width=\textwidth]{L189429-1x1-single-vs-double-sincos-cum-pdf.pdf}
\end{center}
\caption{Effect of single precision in sin cos calculations for
  fringe stopping. Cumulative PDF.}
\label{fig:single-vs-double-cumulative-pdf}
\end{figure}


\begin{experiment}[PASSED 2013-11-20]
{Single precision versus double precision}
{recorded data}
{Can the sin and cos calculations in fringe stopping be done in single
precision?}

\noindent Assess the difference between doing fringe stopping in
single precision versus double precision. Use the double precision as
a proxy for the truth. The main aim is to determine the effect of
rounding error in the sin and cos calculation in the fringe stopping
routines on the visibilities that are produced.

Our conclusion, based on experiments with data set L189429, is that
the RMS fractional error on the visibilities at 1~s and 3.052~kHz
time- and frequency resolution is about $3 \cdot 10^{-5}$. As can be
seen in the right panel of
Fig.~\ref{fig:single-vs-double-diff-ratio-histograms}, the histogram
of deviations in the ratio of double / single precision data is nicely
symmetrical. Given sufficiently large integration times and frequency
ranges, these rounding errors will behave stochastically, leading to
an increase of the noise level of the visibilities. 

The noise contribution due to this problem at the given time- and
frequency resolution is exactly the visibility amplitude multiplied by
$3\cdot10^{-5}$. In case of noise-limited data, this leads to an
imperceptibly small increase of the total noise, because the thermal
noise and numerical noise add in quadrature. For the rounding noise to
become dominant, the total field flux has to be about 1.4~MJy in the
HBA, and much more in the LBA. Keep in mind that this just raises the
noise level, but that the noise still averages down with the square
root of the time-frequency window.

Figure ~\ref{fig:single-vs-double-cumulative-pdf} shows the cumulative
probability density function. That is, the probability that a
visibility at the 1~s~--~3~kHz resolution has a fractional error
\emph{larger} than $x$.
\end{experiment}




\begin{experiment}[PASSED 2013-12-19]
{Setting of static phase/delays per pol and antenna field}
{recorded data}
{Verify that phases and delays are applied per polarization and
  antenna field}

\noindent Use prerecorded data of 3C~196 in the HBA, and Cyg~A in the
LBA. Two minutes of data should suffice. The experiment consists of
the following steps. For both data sets:
\begin{enumerate}
\item Correlate the data, applying the current delay values in use
  for the BG/P.: PASSED 2013-11-01
\item Derive the delays with respect to the mean delay for all core
  stations and verify that it is less than 1~ns for all stations. If
  this is the case, the delays are applied with the correct sign.
\item For one station, and 100~ns to the delay of the Y polarization
  only.
\item For another station, add 70 degrees of phase to the X
  polarization only. 
\item Re-correlate the data again, and fit for delays and phase at
  0~MHz frequency, and verify that the 100~ns and 70$^\circ$ changes
  show up in the appropriate stations and polarizations. PASSED 2013-12-19
\end{enumerate}
\end{experiment}




\begin{experiment}[PASSED 2013-11-15]
{Station UVW coordinates}
{recorded data}
{Verify the J2000 UVW coordinates in the MS}

\noindent For core--core, remote--remote, and ILT--ILT baselines,
verify that the J2000 UVW coordinates in the measurement set are equal
(to within 5 cm) to the UVW coordinates derived from the station's
ITRF positions, transformed using casacore for the epoch of the
observation.

Run the test case \texttt{test\_uvw\_casacore} from the program
\texttt{verify-ms-format.py} listed in appendix
\ref{sec:verify-ms-format}. This was done for L188487, which passed
the test case on 2013-11-15.
\end{experiment}



\begin{figure}
  \begin{center}
    \includegraphics[width=\columnwidth]{L203421-EU-clock-tec.png}
  \end{center}
  \caption{Clock delay and TEC difference between CS002HBA0 and four
    international stations during observation L203395.}
  \label{fig:clock-tec-eu}
\end{figure}


\begin{figure}
  \begin{center}
    \includegraphics[width=\columnwidth]{L203421-EU-scatter-around-360s-avg.png}
  \end{center}
  \caption{Clock delay difference between CS002HBA0 and four
    international stations during observation L203395 after
    subtracting a 360~s running mean.}
  \label{fig:clock-tec-eu-minus-360-avg}
\end{figure}

\begin{experiment}[PASSED 2014-01-31]
{Baseline phases}
{10~h observation on 3C~48 or 3C~147}
{Ensure that there are no sudden jumps in the visibility phases or delays}

\noindent To test if the delays are compensated in a smooth way, and
no major interpolation accidents occur, make plots of the baseline
delays and phase offsets for all international and remote baselines
from a long HBA observation of a compact source.

All delay- and phase structure should be attributable to station
clocks, the ionosphere, and slow moving differences between the
casacore and CALC delay models.

An observation of 3C~48 was conducted on January 31st (L203395). This
observation involved all Dutch and international stations
simultaneously in HBA\_DUAL mode and lasted for about 5 hours. We
fitted for phase 0 offsets and delays for the XX and YY correlations
of between CS002HBA0 and the international stations. DE601, DE604,
DE605, and SE607 provided good data.

Figure~\ref{fig:clock-tec-eu} shows the clock and TEC
differences. Uncertainties for the delays are typically a bit less
than a ns. TEC uncertainties are typically a few mTECU. It is clear
that there are fairly large systematic delay variations of up to
150~ns. The \bgp showed similar behaviour. This is most likely due to
the way we invoke the casacore delay model or problems in the actual
casacore delay model. Because decorrelation due to this problem is
rather benign at  4 channels per sub band resolution, we intend to
solve this after April 1st.

Figure~\ref{fig:clock-tec-eu-minus-360-avg} shows the delays again,
but after subtracting a 360~s running mean. This figure will highlight
any sudden delay jumps, if any. Based on this figure, we can
confidently state that there are no sudden jumps in delay larger than
about half a ns on timescales smaller than 6 minutes. The structure in
the CS002HBA0--SE607HBA baseline is due to the central source being
partially resolved out, and other sources beginning to contribute to
the visibility in a significant way.
\end{experiment}






\chapter{Correlator}

\begin{experiment}
{MS $uvw$ coordinates versus delay compensation $uvw$s}
{recorded data}
{Verify that the $uvw$ coordinates in the MS correspond to
  $\left\langle\vec{u}_j - \vec{u}_i\right\rangle$}

\noindent The $uvw$ coordinates in the MS for a given visibility must
be equal to the mean of the difference between the $uvw$ coordinates
for the stations in J2000, averaged over the integration time of the
visibility. Testing this requires an experiment in which the
correlator logs the offset between a station and CS002LBA in the J2000
frame for at least 2 stations, at \emph{every} 5~$\mu$s time step for
one full integration. Obviously, this is an experiment that may not be
possible to run in real time, but it needs to be done to ensure that
at the very least the delay model and the MS are consistent.

A way to store correlator-computed station UVWs in the MS is currently
[2014-03-11] being worked on. It has not yet been installed.
\end{experiment}




\begin{experiment}[PASSED? 2013-11-22]
{Flux scale 16-bit versus 8-bit mode}
{recorded data}
{Verify that 16-bit and 8-bit data have the same flux scale}

\noindent At the stations, 8-bit data are normalized differently than
16-bit data, leading to a voltage amplitude that is a factor 16 lower
for 8-bit data. The \bgp does not correct for this difference. In
\cobalt, we do want to correct this to ensure a uniform flux density
scale, independent of bit mode.

The experiment is very simple: correlate 8-bit data and 16-bit data,
recorded within minutes of each other, straddling transit of 3C~196,
and compare the visibility amplitudes. They must be the same to within
a couple percent. In the \bgp, they would be different by a factor
256.

This experiment has not yet been done properly, but a preliminary test
with L189429 (16 bit) and L189430 (8 bit) observations of 3C 295 was
done, while 3C 295 was a fair distance away from transit. We ran the
following code on locus004:

{\small
\begin{lstlisting}[language=Python]
from pyrap.tables import table
ms16 = '/data/L189429-reference-single/L189429_SAP000_SB000_uv.MS/'
ms08 = '/data/L189430-reference-single/L189430_SAP000_SB000_uv.MS/'
sel16  = table(ms16).query('SQRT(SUMSQUARE(UVW[1:2])) > 1000.0')
sel08  = table(ms08).query('SQRT(SUMSQUARE(UVW[1:2])) > 1000.0')
data16 = sel16.getcol('DATA')
data08 = sel08.getcol('DATA')
median16 = median(abs(data16)[:,6:-6,0::3]) # avg xx and yy
median08 = median(abs(data08)[:,6:-6,0::3]) # avg xx and yy
print '8 bit: %.3f; 16 bit: %.3f' % (median08, median16)
\end{lstlisting}
}

Which gave ``8 bit: 0.027; 16 bit: 0.020'' as an answer. These numbers
are within 25\% of each other, and certainly not a factor of 256
off. We are therefore fairly confident that this actually works.
\end{experiment}




\begin{figure}
\begin{center}
\includegraphics[width=\columnwidth]{flux-scale-channel-independent.pdf}
\end{center}
\caption{Flux scale is independent of the number of channels.}
\label{fig:flux-scale-channel-independent}
\end{figure}


\begin{experiment}[PASSED 2014-01-29]
{Flux scale independent of nr channels}
{recorded data L189598 Cyg A}
{Ensure that the mean amplitude in each channel is the same,
  independent of the number of channels specified.}

\noindent In the \bgp, Fourier transforms from the time to the
frequency domain were not normalized, leading to a flux scale that is
proportional to the number of output channels. This is clearly not
desirable. For \cobalt, the time to frequency transforms should be
normalized, ensuring a uniform flux scale.

This can be tested with the same data sets that were used for the
station sub band band pass correction
test. Fig.~\ref{fig:flux-scale-channel-independent} shows the
results for 16, 64, and 256 channels.
\end{experiment}





\begin{experiment}[CANCELLED]
{Equivalence to \bgp in 16 bit mode}
{recorded data: 3C~295 all stations}
{Assert that \cobalt output is equal to \bgp output}

\noindent This is one of the most important validations. The
prerecorded data should be fed through both \cobalt \emph{and} the
\bgp. The output data sets are subsequently subtracted. Because the
Fourier transforms in \cobalt may at some point all be normalized to
ensure that the flux scale is independent of the number of channels,
one may need to apply a constant multiplication factor between \cobalt
and the \bgp before subtracting the data sets.

Any deviations must be minor
and consistent with numerical noise, expected to be at a level of
about $10^{-5}$ of the visibility amplitude. The difference must also
be un-biased. If \cobalt passes this test, we have conclusively shown
that \cobalt is no worse than the \bgp.

Because the BG/P apparently can not digest offline UDP packets
anymore, we cannot do this experiment and will have to independently
verify \cobalt's functionality and performance.
\end{experiment}







\begin{figure}
\begin{center}
\includegraphics[width=0.8\columnwidth]{cobalt-performance-subband-completeness-202756-202772.pdf}
\end{center}
\caption{Fraction of sub bands without any data loss in the final
  Measurement Sets.}
\label{fig:data-completeness-correlator}
\end{figure}


\begin{figure}
  \begin{center}
    \includegraphics[angle=0,width=\columnwidth]{correlator-input-loss-as-function-of-field.pdf}
  \end{center}
  \caption{Data loss at the input of Cobalt per antenna field as a
    function of the total number of antenna fields in the
    observation. These are the statistics for three days in a period
    in which we were finalizing the input optimizations. Note the
    scale of the colorbars. It goes down by a factor 10 each day.}
  \label{fig:correlator-input-loss}
\end{figure}


\begin{figure}
  \begin{center}
    \includegraphics[angle=0,width=1.1\columnwidth]{correlator-input-loss-as-function-of-field-20140313.pdf}
  \end{center}
  \caption{Data loss at the input of Cobalt per antenna field as a
    function of the total number of antenna fields in the
    observation. This is an enlargement of the third panel in Fig.~\ref{fig:correlator-input-loss}}
  \label{fig:correlator-input-loss-20140313}
\end{figure}


\begin{experiment}[PASSED 2014-03-13]
{Correlator capacity}
{real time data}
{Find the maximum correlator capacity}

\noindent In 8 bit mode with 488 sub bands, in HBA\_DUAL
configuration, with 256 channels per sub band and 1 second integration
time, conduct a series of 1 minute observations in which the number of
correlated fields is increased every time until the correlator can not
keep up anymore.

Estimate, from the processing power / network capacity that is used as
a function of the number of stations, at which point the correlator
can not keep up anymore. Once all stations are being correlated, one
can try to decrease the integration time in steps of a factor of two
until data writing breaks down, to establish the maximum obtained
throughput to CEP2.

We tested 48 to 62 antenna fields (2014-01-29). SAS IDs
202756..202772. We find that Cobalt's percentage of complete data sets
(Fig.~\ref{fig:data-completeness-correlator}) roughly follows the
shape of the \bgp's curve, but looses a bit more, and behaves more
erratically. At the data rates of typical observations (4 Gbit/s),
however, we do not loose data. Although the erratic behaviour must be
solved, the performance is adequate for regular operations.

We also tested the input loss before and after optimizing the input
section of Cobalt, after pretty much having rewritten the entire
initial data distribution before correlating. We observed in 8~bit
mode, using 488 sub bands, 64 channels per sub band, and 1~s
correlator integration time. The result is shown in
Fig.~\ref{fig:correlator-input-loss}. We have shown that we typically
loose less than one percent of data per station on 120~s long
observations. Most loss is incurred in the first 5--15 seconds. The
international stations drop way more data, but those data simply do
not arrive at Cobalt due to issues in the long haul network
connections towards those stations. These connection problems have
been resolved after this experiment was conducted. The left hand graph
is before basic optimization, the right hand graph is the current
situation. That panel is enlarged in
Fig.~\ref{fig:correlator-input-loss-20140313}.
\end{experiment}





\begin{experiment}[PASSED 2014-03-14]
{Long, deep integration}
{real data (12h integration on 3C~48 or 3C~147)}
{Establish long term stable operation of \cobalt}

\noindent Observe 12~h of 3C~48 HBA\_DUAL\_INNER data in 8-bit mode
with all Dutch LOFAR stations. This is basically an EoR type
experiment. Flag the data and calibrate with the best source model
there is for this field, obtainable from V.~Pandey. If we reach the
thermal noise on the inner 8~h of data -- expected to be around
$100$--$150\ \mu$Jy/beam. -- and there are no artifacts visible in the
residual map, the correlator pipeline is good enough.


We have taken a 5h EoR type observation of 3C~196, which showed
correlator capacity problems that had appeared during the
2014-02-03 stop day. These data curently reside on the EoR cluster and
still need to be imaged.


On 2014-03-13, we performed an 11h 8 bit observation of 3C~196 with
all stations, except DE604, in HBA\_DUAL\_INNER mode. That is, we used
69 antenna fields. The observation ran fine and lost no data at the
output. Input losses were benign, and most likely due to data not
arriving at \cobalt in the first place. The only stations that lost
0.01~\% or more of their input data were:

\begin{description}
  \item[CS004HBA0] 0.01\%
  \item[CS004HBA1] 0.01\%
  \item[CS401HBA0] 0.03\%
  \item[CS401HBA1] 0.03\%
  \item[DE601HBA] 2.65\%
  \item[DE602HBA] 0.06\%
  \item[DE603HBA] 32.57\%
  \item[DE605HBA] 2.63\%
  \item[FR606HBA] 2.64\%
  \item[SE607HBA] 38.98\%
  \item[UK608HBA] 2.52\%.
\end{description}

The input losses were likely due to data that never even arrived at \cobalt.

The data are currently [2014-03-14] at the EoR cluster, and will be
imaged by V.~Pandey.
\end{experiment}










\chapter{Beamformed modes}


\begin{figure}
  \begin{center}
    \includegraphics[width=\columnwidth]{dailyimage-20140206-cobalt-offline-pulsar.png}
  \end{center}
  \caption{Detection of LGM-1 with incoherent and coherent beam
    forming by \cobalt in offline mode.}
  \label{fig:lgm-1}
\end{figure}


\begin{figure}
  \begin{center}
    \includegraphics[width=0.48\columnwidth]{cs-snr-vs-stations-2014-07-08.png}%
    \includegraphics[width=0.48\columnwidth]{is-snr-vs-stations-2014-07-08.png}
  \end{center}
  \caption{Coherent Stokes (left) and incoherent Stokes (right)
    signal-to noise ratio as a function of the number of stations in
    the (in)coherent sum. The SNR falls well below the theoretically
    optimal (green) curves. Cause is being investigated.}
  \label{fig:cs-is-snr-scaling}
\end{figure}

\begin{experiment}[IN PROGRESS]
{Coherent stokes}
{recorded data: PSR~B0329+54}
{Verify that coherent addition increases SNR linearly}

\noindent Take the PSR~B0329+54 test data sets, and coherently add 1,
2, 4, 8, 16, 32, and 46 antenna fields. Do this in 8-bit and 16-bit
mode. The SNR increase must be linear in the number of stations, and
the flux scale the same in both bit modes.

The data must be processed using the pulsar pipeline and the PRESTO
package. Repeat this in Stokes $I$, $IQUV$, and for complex
voltages. Use 1, 16, and 64 channels.

We have shown that coherent and incoherent addition work in offline
mode in the daily image of [2014-02-14] (see
Fig~\ref{fig:lgm-1}). On-line addition works too, however, as
Fig.~\ref{fig:cs-is-snr-scaling} shows, its signal to noise ratio
(SNR) falls well below the theoretically expected curve. The cause may
be the station-delay calibration table at \cobalt, something at the
stations, local interference, or a problem in \cobalt's addition
kernels. These things are currently (2014-08-29) under investigation.
\end{experiment}



\begin{experiment}[IN PROGRESS]
{Incoherent stokes}
{recorded data: PSR~B0329+54}
{Verify that incoherent addition increases the SNR $\propto \sqrt{N}$}

\noindent \ldots where $N$ is the number of stations. Take the
PSR~B0329+54 test data sets, and coherently add 1, 2, 4, 8, 16, 32,
and 46 antenna fields. Do this in 8-bit and 16-bit mode. The SNR
increase must be proportional to the square root of the number of
stations, and the flux scale must be the same in both bit modes.

The data must be processed using the LOFAR pulsar pipeline. Repeat
this in Stokes $I$, $IQUV$, and for complex voltages. Use 1, 16, and
64 channels.

We have shown that coherent and incoherent addition work in offline
mode in the daily image of [2014-02-14] (see
Fig~\ref{fig:lgm-1}). On-line addition works too, however, as
Fig.~\ref{fig:cs-is-snr-scaling} shows, its signal to noise ratio
(SNR) falls well below the theoretically expected curve. The cause may
be the station-delay calibration table at \cobalt, something at the
stations, local interference, or a problem in \cobalt's addition
kernels. These things are currently (2014-08-29) under investigation.
\end{experiment}


\begin{figure}
  \begin{center}
    \includegraphics[width=\columnwidth]{L212698_combined-FE-2014-03-20-cobalt.png}
  \end{center}
  \caption{One of the first successful poor man's Fly's Eye
    observations with \cobalt.}
  \label{fig:fe-hba}
\end{figure}

\begin{experiment}[PASSED 2014-04-24]
{Fly's eye}
{recorded data: PSR~B0329+54}
{Ensure we can separately record data from individual stations}

\noindent Set up parallel observations of 1, 2, 4, 8, 16, 32, and 46
antenna fields. process the resulting data and recover the pulsar SNR
per station. Do this only in 16 bit mode for Stokes $IQUV$.

We have done several poor-man's fly's eye observations (all stations
looking in the same direction), which is a useful system debugging
tool. This seems to work d.d. [2014-03-13] for 69 antenna fields in
16-bit mode. Proper fly's eye observing (every station observing a
different direction) is not yet possible [2014-03-14].

In fact, this can be done without resorting to parallel
observations. Specifying ``poor-man's Fly's Eye'' observations (all
stations pointing in the same direction) works the same way as with
BG/P. See Fig.~\ref{fig:fe-hba} for first results. A full fly's eye
observation (all stations in independent directions) must (and can) be
specified by multiple parallel observations.
\end{experiment}



\begin{experiment}[CANCELLED]
{Equivalence to BG/P in 16 bit mode}
{recorded data: PSR~B0329+54}
{Assert that \bgp and \cobalt data are equivalent}

\noindent The prerecorded data should be fed through both \cobalt
\emph{and} the \bgp. The output data sets are subsequently
subtracted. Because the Fourier transforms in \cobalt may at some
point all be normalized to ensure that the flux scale is independent
of the number of channels, one may need to apply a constant
multiplication factor between \cobalt and the \bgp before subtracting
the data sets.

Any deviations must be minor and consistent with numerical noise,
expected to be at a level of about $10^{-5}$ of the timeseries
amplitude. The difference must also be un-biased. If \cobalt passes
this test, we have conclusively shown that \cobalt is no worse than
the \bgp.
\end{experiment}




\begin{experiment}
{Coherent dedispersion}
{synthetic and recorded data}
{Verify that coherent dedispersion works in HBA and LBA.}

\noindent Using synthetic data of a highly dispersed pulse, verify
that correcting for the exact dispersion measure yields the shape and
amplitude of the pulse that went in.

Using prerecorded station data of a highly dispersed pulsar, perform
coherent stokes beam forming, applying coherent dedispersion, and
verify that the signal-to-noise ratio and pulse profile in the output
correspond with what the pulsar group expects.

This functionality will be implemented during regular operations.
\end{experiment}






\begin{experiment}[PASSED]
{Coherent beam forming capacity}
{real time data}
{Find the maximum number of coherent beams}

\noindent In 8 bit mode with 488 subbands, in HBA\_DUAL configuration,
with 16 channels per sub band and no time integration, conduct a
series of 5 minute coherent stokes observations in which the number of
tied array beams is increased every time until the correlator can not
keep up anymore. Beam form the entire core in HBA\_DUAL mode with the
following tied array beam numbers: 1, 2, 4, 8, 16, 32, 64, 128, 256,
512 and 17, 31, 67, 127, 257, 509.

Do not use coherent dedispersion.

We have been probing the maximum capacity using tied array beam
rings. We can do 5 tied array rings with 488 sub bands and all
superterp HBA fields, loosing about 2\% of data in that case. 4 tied
array rings plus 12 separate tied array beams can be done with
37~Gbit/s data output rate, dropping only 0.8\% of the data enroute to CEP2.

For the full core, we can do 6 tied array beams with 162 sub bands
each. Beamformer performance is a function of approcimately 7
variables, making this a very hard parameter space to fully
explore. We can computationally support al types of pulsar
observations that were requested for LOFAR Cycle 2.
\end{experiment}




\begin{experiment}
{Coherent dedispersion capacity}
{real time data}
{Find the maximum dispersion that can be handled in a single tied
  array beam}

\noindent For both LBA and HBA\_DUAL, in 8-bit mode with 488 sub bands
of 16 channels each, and one tied array beam, do coherent
dedispersion, increasing the dispersion measure in factors of 2,
beginning at 1, until \cobalt runs out of memory.

Coherent dedispersion will be implemented later.
\end{experiment}




\begin{experiment}[IN PROGRESS]
{Long, deep integration}
{real time data}
{Establish long term stability}

\noindent Observe a weak pulsar and its environment using the maximum
number of tied array beams in 8 bit mode with 46 antenna fields for
several hours. The system may not crash, and must yield thermal noise
limited data.

The beamformer is stable over several hours of observing. However, the
coherent stokes sensitivity with the core is currently (2014-08-29)
subpar. Cause is yet unknown. It may be the delay calibration table,
something in the coherent stokes kernel, or something with the
stations themselves. This is actively being investigated.
\end{experiment}

\begin{figure}
  \begin{center}
    \includegraphics[width=\columnwidth]{resids-postfit-Cobalt-points-fitted.png}
  \end{center}
  \caption{Timing residuals on J0034-0534 after jointly fitting BG/P
    and Cobalt data.}
  \label{fig:timing-residuals}
\end{figure}


\begin{experiment}[PASSED 2014-05-28]
{Millisecond pulsars}
{real time data}
{Establish absolute time}

\noindent Observe a group of known millisecond pulsars to find out if
the time scale in the output files is reproducably close enough to
UTC.

Timing residuals on J0034-0534 are approximately 24~$\mu$s RMS and
comparable to BG/P era. See Fig.~\ref{fig:timing-residuals}.
\end{experiment}









\chapter{Parallel observations}

\begin{experiment}[PASSED 2014-06-17]
{Multiple simultaneous identical pipelines}
{real time data}
{Ensure we can run multiple simultaneous pipelines at all}

\noindent \cobalt's fly's eye mode will be implemented with multiple
parallel observations where each antenna field take part in at most
one observation. This is obviously the simplest case.

In this experiment, run multiple simultaneous coherent stokes
observations of the same pulsar in 8-bit HBA\_DUAL mode. Every
observation uses one antenna field.  Use progressively more antenna
fields in approximate powers of two:
\begin{itemize}
\item superterp stations (6 stations, 12 fields)
\item half of the core stations (12 stations, 24 fields)
\item all core stations (24 stations, 48 fields)
\item all Dutch stations (38 stations, 62 fields)
\item all stations       (46 stations, 70 fields)
\end{itemize}

Repeat this experiment for the correlator pipeline by executing
parallel observations with the core only, remote stations only, and
international stations only.

We have, technically, not strictly tried this experiment, but
discovered on June 17, that we can schedule and run arbitrary
observations simultaneously, using different start and stop times,
provided that the station sets do not overlap.
\end{experiment}



\begin{experiment}[IN PROGRESS]
{Multiple different, simultaneous  pipelines, overlapping data}
{real time data}
{Run interferometric and beam formed observations at the same time}

\noindent Produce correlated data and beam formed data from the
\emph{same} set of stations.

Code is ready for commissioning as of 2014-08-29.
\end{experiment}





\begin{experiment}[PASSED 2014-06-17]
{Truly independent scheduling}
{real time data}
{Independently schedule and run different observations}

\noindent Run a set of independent observations in different beam
forming modes and interferometric modes, that may not start and end at
the same time.

This is mostly a long term wish, testing the observatory software more
that \cobalt itself.

It works! On June 17, we scheduled and ran arbitrary observations
simultaneously, using different start and stop times, provided that
the station sets do not overlap. This possibility has since been used
regularly for solar observations and interplanetary scintillation
experiments.
\end{experiment}







\chapter{Observatory software integration}


\begin{experiment}[PASSED 2013-11-01]
{Switch data streams to \cobalt or \bgp}
{real data}
{Ensure easy switching between the two correlators}

\noindent The switching is done by regenerating the configuration
files for the stations. This is done by the createFiles script. This
works beautifully for the Dutch stations. Switching between BG/P and
Cobalt operations takes about 2 minutes now.

International stations still have routing issues, but I consider that
separate from this experiment. In fact, most of those issues have been
solved on 2013-11-26/27, and the final remaining connection issues to
the international stations were resolved at the end of spring 2014.
\end{experiment}




\begin{experiment}[PASSED 2014-03-14]
{System validation observations}
{real data}
{Verify seamless integration into observatory software}

\noindent Run the standard suite of system validation observations
from MoM through the Scheduler until all validation plots have
appeared. At no point should it be necessary to specify anything
\cobalt specific. This assumes the switch to \cobalt has already been
made by the script used in the first experiment.

Finally, export the raw data of this validation run to the LTA.

On 2014-03-14 we have performed a successful system validation
run. We have also ingested raw data from a previous, incomplete run
into the LTA. The data and metadata of those observations appeared
fine in the LTA.
\end{experiment}





\begin{experiment}[PASSED 2014-04-24]
{CEP-2 pipeline processing}
{real data}
{Verify that current processing pipelines are not broken}

\noindent Run observations with attached pipelines. Observations need
not be longer than 10--20 minutes. Use the following combinations:
\begin{itemize}
\item LBA\_INNER, flagging and demixing [PASSED 2014-01-27]
\item HBA\_DUAL\_INNER, flagging and averaging [PASSED 2014-01-27]
\item HBA\_DUAL, target/calibrator pipelines
\item HBA\_DUAL\_INNER, MSSS observation all the way to automatic imaging
\end{itemize}

Finally, export the processed data to the LTA.

Raw data as well as processed data has been exported to the LTA
successfully. We run all pipelines that ran for BG/P data successfully
in production as of 2014-04-24.
\end{experiment}


\chapter{Conclusions}

\cobalt is sufficently well tested to use in regular observations,
although some issues remain, primarily in the beam former
pipeline. The highest priority issues to solve in the near future are

\begin{itemize}
\item poor scaling of coherent- and incoherent Stokes sensitivity as a
  function of number of input stations;
\item residual station band pass in beam formed modes;
\item delay model errors, particularly affecting long baselines.
\end{itemize}


\appendix


\chapter{verify-ms-format.py}
\label{sec:verify-ms-format}

{\scriptsize
\lstinputlisting[language=Python]{verify-ms-format.py}
}


\addcontentsline{toc}{chapter}{Bibliography}
\bibliographystyle{aa}
\bibliography{cobalt}

\vspace{12cm}


\end{document}
