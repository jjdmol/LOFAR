\documentclass[11pt,twoside]{article}  % Leave intact
\usepackage{adassconf}

% If you have the old LaTeX 2.09, and not the current LaTeX2e, comment
% out the \documentclass and \usepackage lines above and uncomment
% the following:

%\documentstyle[11pt,twoside,adassconf]{article}

\begin{document}   % Leave intact

\paperID{O01-4}
% If your title is so long as to fill the page header when you print it,
% then please supply a short form as a \titlemark.
\title{The LOFAR Global Sky Model: Some Design Challenges}
%\titlemark{The LOFAR Global Sky Model}

\author{Oleg M. Smirnov, Jan E. Noordam}
\affil{Netherlands Foundation for Research in Astronomy (ASTRON)}

\contact{Oleg Smirnov}
\email{smirnov@astron.nl}

\paindex{Smirnov, O.M.}
\aindex{Noordam, J.E.}     % Remove this line if there is only one author

\authormark{Smirnov \& Noordam}

\keywords{telescopes: LOFAR, astronomy: radio, calibration: selfcal,
calibration: polarization, interferometry: radio, sky model}

\begin{abstract}          

The LOFAR Global Sky Model (GSM) will be an all-sky database of some 100
million objects, with flux \& polarization measurements in the 20--200 MHz
range. The primary function of the GSM is to support LOFAR calibration and data
reduction. The GSM is expected to provide a model of all sufficiently  bright
sources in any given field, having enough detail and precision to calibrate and
subtract these sources and yield residual images of the faint background. The
GSM is expected to be continuously updated and refined during LOFAR operation
in a ``closed loop'' of sorts. The GSM is also a valuable stand-alone data
product that can be made compatible with the VO.

The instrumental characteristics of LOFAR pose large challenges to GSM design,
some of them unique even in the field of very large catalogues. Besides sheer
size, this includes highly complex source models (thus making for a very
non-uniform database structure), stringent performance requirements for
operational use, the need to update source models operationally, and various
data management issues. This paper will focus on some of these challenges,
discuss our approaches to dealing with them, and present a prototype GSM being
developed for the LOFAR Pilot Selfcal System (PSS).

\end{abstract}

\section{Introduction}

\htmladdnormallinkfoot{LOFAR}{http://www.lofar.org} is a a large, distributed
radio telescope being designed by an international consortium (ASTRON, ATNF,
MIT Haystack, NRL). Some architectural features of the current design are:

\begin{itemize}
\item 20--200 MHz range;
\item $>100$ phased array stations combined into an aperture synthesis array;
\item Log-spiral configuration with a dense central core, largest baseline is
$\approx$300 km.
\end{itemize}

The unique instrumental characteristics of LOFAR pose new challenges to
calibration. This has led us to formulate a requirement for a Global Sky Model
(GSM):

\begin{itemize}

\item The GSM itself, being a very large source catalog updated over the
lifetime of the telescope, is one of the main LOFAR deliverables. As such, it
will be made available in the VO.

\item Calibration of LOFAR for any observation requires reference sources from
all over the sky, due to the extensive sidelobes of the instrument.

\item The instrument will have a very wide field of view, and will utilize
multiple beams. Thus, full data reduction will require {\em facet imaging},
which needs a global grid of reference sources to align the individual images
in position and flux.

\end{itemize}

\section{GSM: the next step in radio sky models}

The self-calibration algorithm employed in radio astronomy is simple in
essence, but extremely difficult in the details. The basic selfcal step is
predict sky -- apply instrumental effects -- compare to observed data. The
current record for dynamic range with selfcal is $10^6$, achieved at the WSRT
(***WSRT). The LOFAR design target is $10^8$. A proof of concept study will be
carried out using WSRT data, which should be able to go up to $10^7$, given a
sophisticated enough calibration approach.

One of the main limiting factors in the dynamic range of selfcal is accuracy of
the predict step. An accurate predict requires a sufficiently complex \&
flexible flux source representation:

\begin{itemize}

\item The traditional sky model -- still widely used today by packages such as
\htmladdnormallink{AIPS}{http://www.aoc.nrao.edu/aips/} -- is a collection of
CLEAN components. This is well-suited to representing point sources, but fails
to model extended emission with any sort of accuracy. Various refinements of
CLEAN (see Cornwell 2004 for an overview), have been developed over the years
to address this shortcoming.

\item \htmladdnormallink{NEWSTAR}{http://www.astron.nl/newstar/}, the WSRT data
reduction package, introduced a parametrized source representation ($IQUV$
fluxes, rotation measure, spectral index), as well as elliptical gaussians to
model extended sources.

\end{itemize}

LOFAR (and other upcoming instruments such as SKA) requires us to take the next
step. This includes more sophisticated source models, in particular spatially
extended sources, as well as time and frequency variability of all source
parameters. 

\subsection{Some features of the LOFAR GSM}

\begin{itemize}
\item Expected size: 100 million objects. All-sky coverage.
\item Fluxes and polarizations in the 20--200 MHz range.
\item Very rich with spatially extended sources. Needs sophisticated
parametrizations, with all parameters potentially being functions of time and
frequency.
\item Must support a variety of object morphologies: point sources, extended
sources as shapelets, pixons, CLEAN components, images, et al.  Morphology can
change qualitatively with frequency.
\item Initially populated from existing catalogues and updated over the lifetime
of the instrument.
\item One of the two main deliverables of LOFAR.
\end{itemize}

\subsection{The GSM lifecycle}

Operationally, the GSM software must support the following lifecycle:

\begin{enumerate}
\item For any observation, a required subset is extracted into a Local Sky Model
(LSM);
\item Calibration is performed, the LSM is updated with new calibration
solutions and possibly new sources;
\item Changes are optionally committed back to the GSM;
\item After imaging (which produces the data products), images of faint sources
may also be added to the GSM.
\end{enumerate}

\section{Data representation in the GSM}

The question of data representation has emerged as one of the central issues of
GSM design. This representation must be sufficiently flexible to capture
sufficient detail for calibration and imaging. Because of the 
GSM--calibration--GSM ``closed loop'', the notion of parametrized sources must
be explicitly supported by the GSM itself. Thus, what is really required is a
non-uniform and extensible source representation.

\subsection{MeqTrees}

{\em MeqTrees} (Measurement Equation Trees) are a mechanism being developed in
the LOFAR Pilot Selfcal System (PSS). A MeqTree\footnote{Strictly speaking,
these are not limited to tree structures, but are a more general type of graph
-- the directed acyclic graph (DAG); the name {\em MeqTree} is now entrenched
for historical reasons.} corresponds to a mathematical expression. The leaf
nodes represent parameters and data sources, while a root node represent 
derived expressions, such as the predicted visibility of a source in a given
direction.

A MeqTree can recursively evaluate its expression, and estimate partial
derivatives w.r.t. specific parameters. Thus, a MeqTree can be employed to
iteratively solve an equation for a set of parameters. All parameters in the
tree ({\em MeqParms\/}) are polynomials (and in the future, possibly other
smooth functions) of frequency and time, so the actual solvables are the
individual polynomial coefficients.

Since any mathematical equation can be represented in tree form, PSS-4 (the
current development cycle of PSS) should be able to solve for arbitrary
measurement equations. Operationally, the system uses a scripting layer (the
Glish language of
\htmladdnormallink{AIPS++}{http://aips2.nrao.edu/docs/aips++.html})  to set up
trees, and a fast C++ kernel to evaluate and solve them.

\subsection{The GSM as a collection of trees}

MeqTrees provide a perfect answer to the issue of GSM data representation. To
predict the sky, we need to be able to predict the visibility contribution of
each source in a particular direction. This, however, may need to be computed
differently for different types of sources. By representing a source (or rather,
the components of a source, such as ``the Stokes $I$ flux in a specific
direction for a specific interferometer'') as a MeqTree, we reap a number of 
benefits:

\begin{itemize}

\item {\bf Polymorphism:} all sources appear the same way (that is, as the root node of
a tree) both to the user and to the calibration system, regardless of specific
internal representation.

\item {\bf Extensibility:} source structure can be represented and paremetrized to
any level of complexity

\item {\bf Parellelization:} trees allow for fine-grained parallelizm, which is very
important considering the immense computing requirements of LOFAR calibration.

\item {\bf Rapid experimentation:} trees can be torn down and rebuilt at run-time. This
allows for quick experiments with alternative source representarions.

\end{itemize}

Of course, this approach also incurs a number of trade-offs:

\begin{itemize}

\item {\bf Computational overhead:} trees require more complicated housekeeping and
data management compared to a ``hard-wired'' source reprsentation.

\item {\bf DB complications:} this type of structure is rather difficult to represent
in a traditional relational database.

\end{itemize}

\section{Development plan}

An initial version of the LOFAR GSM (GSM-0), is being developed as part of
the PSS-4 cycle. This is meant as proof-of-concept implementation, thus we have
set the following modest goals:

\begin{itemize}

\item A homogenous source representation borrowed from NEWSTAR, with a fixed
set of source parameters: RA/Dec, $IQUV$ fluxes at a reference frequency,
spectral index, rotation measure. Extended sources will be modeled  by
elliptical gaussians requiring three additional parameters: major/minor axis
size, and orientation. 

\item All parameters can be 2D polynomials of time and frequency.

\item GSM-0 is stored in a simple flat table. In the background, we will work
towards a full database implementation of MeqTrees.

\item Internally, PSS-4 will use fully-featured MeqTrees. These will be
constructed on-the-fly in the GSM-0 code, using parameters from the GSM table.
Thus, we will be able to test calibration will fully featured MeqTrees, but the
structure of the trees will be hardwired in GSM-0.

\end{itemize}





%-----------------------------------------------------------------------
%			      References
%-----------------------------------------------------------------------
% List your references below within the reference environment
% (i.e. between the \begin{references} and \end{references} tags).
% Each new reference should begin with a \reference command which sets
% up the proper indentation.  Observe the following order when listing
% bibliographical information for each reference:  author name(s),
% publication year, journal name, volume, and page number for
% articles.  Note that many journal names are available as macros; see
% the User Guide listing "macro-ized" journals.   
%
% EXAMPLE:  \reference Hagiwara, K., \& Zeppenfeld, D.\  1986, 
%                Nucl.Phys., 274, 1
%           \reference H\'enon, M.\  1961, Ann.d'Ap., 24, 369
%           \reference King, I.\ R.\  1966, \aj, 71, 276
%           \reference King, I.\ R.\  1975, in Dynamics of Stellar 
%                Systems, ed.\ A.\ Hayli (Dordrecht: Reidel), 99
%           \reference Tody, D.\  1998, \adassvii, 146
%           \reference Zacharias, N.\ \& Zacharias, M.\ 2003,
%                \adassxii, \paperref{P7.6}
% 
% Note the following tricks used in the example above:
%
%   o  \& is used to format an ampersand symbol (&).
%   o  \'e puts an accent agu over the letter e.  See the User Guide
%      and the sample files for details on formatting special
%      characters.  
%   o  "\ " after a period prevents LaTeX from interpreting the period 
%      as an end of a sentence.
%   o  \aj is a macro that expands to "Astron. J."  See the User Guide
%      for a full list of journal macros
%   o  \adassvii is a macro that expands to the full title, editor,
%      and publishing information for the ADASS VII conference
%      proceedings.  Such macros are defined for ADASS conferences I
%      through XI.
%   o  When referencing a paper in the current volume, use the
%      \adassxii and \paperref macros.  The argument to \paperref is
%      the paper ID code for the paper you are referencing.  See the 
%      note in the "Paper ID Code" section above for details on how to 
%      determine the paper ID code for the paper you reference.  
%
\begin{references}

AIPS, AIPS++, NEWSTAR papers?
\reference WSRT dynamic range
\reference Cornwell, T.J.\ 2004, \adassxiii, \paperref{O2-2}

\end{references}

% Do not place any material after the references section

\end{document}  % Leave intact
